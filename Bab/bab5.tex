\chapter{Implementasi dan Pengujian}
\label{chap:implementasi-pengujian}

\section{Implementasi}

\subsection{Lingkungan Implementasi Perangkat Keras}
    Aplikasi pendukung sistem ujian Oxam akan dijalankan dengan menggunakan \textit{server} milik lab. 
    Spesifikasi perangkat keras yang digunakan adalah sebagai berikut:
    \begin{itemize}
        \item \textit{Processor}:\\ Intel\textsuperscript{\textregistered} Xeon\textsuperscript{\textregistered} CPU E5-2603 v4 @ 1.70GHz
        
        \item \textit{RAM}:\\ 1$\times$8GB DDR4 @ 2400MHz 
        
        \item \textit{Storage}:\\ 2$\times$HPE 1TB 6G SATA 
            7.2K rpm LFF (3.5in) MB001000GWFGF Hard Drive
    \end{itemize}
    
    \subsection{Lingkungan Implementasi Perangkat Lunak}
    Aplikasi pendukung sistem ujian Oxam akan diimplementasi dengan spesifikasi berikut:
    \begin{itemize}
    % Perangkat lunak kudu Runtime Environment:
    %  - Docker (php...)
        \item \textit{Runtime Environment}:
            Docker versi 19.03.8, \textit{build} \texttt{afacb8b7f0}, dengan konfigurasi kontainer:
            \begin{itemize}
                \item backend\\
                    \texttt{php:7.3.8-apache}\\
                    Berkas \texttt{Dockerfile} dapat dilihat pada lampiran %TODO: Lampiran
            \end{itemize}
        \item Bahasa Pemrograman: PHP v7.3, JavaScript (Babel Preset React).
        \item Basis data: MySQL versi 5.7
        \item CI/CD: GitLab CI
        \item \textit{Library} lain yang digunakan:
            \begin{itemize}
                \item Backend: \texttt{chez14/f3-ilgar} \\
                    \textit{Library} ini digunakan untuk membuat database dengan merepresentasikan setiap update
                    menjadi sebuah paket migrasi. Pada aplikasi ini, \texttt{chez14/f3-ilgar} digunakan untuk
                    membantu mengelola basis data.
                \item Backend: \texttt{lcobucci/jwt} \\
                    \textit{Library} ini bertugas untuk membuat token dalam bentuk JWT, melakukan validasi, dan
                    menekstrak data dari token tersebut. \texttt{lcobucci/jwt} digunakan pada saat otentikasi
                    untuk API.
                \item Backend: \texttt{psx/openssl} \\
                    \texttt{psx/openssl} adalah \textit{library} yang membungkus fungsi OpenSSL yang tersedia
                    pada PHP. \textit{Library} ini akan membuat kode menjadi lebih bersih karena \textit{overhead}
                    harus ditulis pada saat menginisialisasi OpenSSL menjadi otomatis pada saat kelas dari 
                    \textit{library} ini diinstansiasi.
                    \textit{Library} ini bertugas untuk membantu \textit{library} \texttt{lcobucci/jwt} menghasilkan kunci
                    asimetrik dengan algoritma RSA atau ECDSA. 
                \item Backend: \texttt{respect/validation} \\
                    \textit{Library} ini bertugas untuk melakukan validasi yang cukup kompleks. \textit{Library}
                    ini dapat diturunkan menjadi fungsi validasi yang lain. Pada Aplikasi ini, \textit{library} ini
                    digunakan untuk membantu validasi data yang diberikan dari klien lewat subsistem
                    Frontend.
                \item Backend: \texttt{adldap2/adldap2} \\
                    \textit{Library} \texttt{adldap2/adldap2} adalah \textit{wrapper} dari fungsi LDAP yang PHP sediakan.
                    \textit{Libray} ini nantinya akan bertugas membantu Backend melakukan \textit{query} 
                    pada LDAP milik Lab Komputasi untuk \textit{resolve} \textit{username} login peserta
                    menjadi nama lengkap.
                \item Backend: \texttt{xfra35/f3-cron} \\
                    \textit{Library} ini bertugas untuk melakukan penjadwalan \textit{cron} yang terdapat
                    pada sistem. Sistem operasi cukup melakukan pemanggilan pada \textit{endpoint} yang disediakan
                    oleh \textit{library} ini, lalu \textit{library} ini akan menjadwalkan cronjob yang ada.
                    Pada aplikasi ini, \textit{library} ini digunakan untuk mengatur \textit{cronjob} pengiriman
                    laporan berkas jawaban ke dosen koordinator.
                \item Backend: \texttt{phpmailer/phpmailer} \\
                    \texttt{phpmailer/phpmailer} bertugas untuk membantu pengiriman email melalui protokol SMTP.
                    Pada aplikasi Oxam, \texttt{phpmailer/phpmailer} bertugas untuk membantu mengirimkan email
                    laporan berkas jawaban ke dosen koordinator.
                \item Backend: \texttt{monolog/monolog} \\
                    \textit{Library} ini bertugas untuk membuat kelas yang digunakan untuk mencatat log yang
                    terjadi pada sistem. Pada aplikasi ini, \texttt{monolog/monolog} digunakan untuk membuat log
                    tentang ujian, seperti peserta mengunggah jawaban, dan seterusnya.
                \item Frontend: \texttt{@fortawesome/*} \\
                    \textit{Library} ini berisi ikon-ikon yang digunakan untuk merepresentasikan fungsi
                    tombol dalam bentuk gambar. Pada aplikasi ini, \texttt{@fortawesome/*} digunakan pada
                    banyak halaman untuk mempermudah pendesainan bebrapa tombol menjadi lebih ringkas.
                \item Frontend: \texttt{axios} \\
                    \textit{Library} ini digunakan untuk melakukan permintaan \textit{ajax} dengan penanganan 
                    \textit{event} berbentuk \textit{Promise}. Pada aplikasi ini, \texttt{axios} digunakan untuk
                    melakukan komunikasi antara subsistem Frontend dan Backend. \textit{Library} ini kemudian
                    diturunkan dengan menambahkan beberapa \textit{event} khusus seperti menambahkan
                    token otentikasi pada saat sistem frontend mendeteksi token pada \textit{cookie}
                    peramban pengguna.
                \item Frontend: \texttt{bootstrap} \\
                    \textit{Library} ini digunakan untuk membuat tampilan web menjadi lebih seragam, lebih
                    mudah diprediksi, dan konsisten antar perangkat. Pada aplikasi ini, \texttt{bootstrap}
                    digunakan pada seluruh halaman aplikasi untuk membuat \textit{look-and-feel} yang seragam,
                    responsif, dan rapi.
                \item Frontend: \texttt{date-fns} \\
                    \textit{Library} ini bertugas untuk melakukan manipulasi tanggal, \textit{parsing} dan 
                    \textit{formatting}. Pada aplikasi ini, \texttt{date-fns} digunakan untuk \textit{parsing}
                    tanggal.
                \item Frontend: \texttt{js-file-download} \\
                    \texttt{js-file-download} berfungsi untuk memicu \textit{file picker} untuk menyimpan berkas.\\
                    \texttt{js-file-download} akan menerima sebuah \textit{blob} binary dari berkas, lalu meminta
                    peramban untuk menampilkan pemilih berkas. Pada aplikasi ini, \texttt{js-file-download}
                    digunakan untuk membantu peserta, dosen koordinator, dan admin mengunduh berkas jawaban,
                    dan \textit{script}.
                \item Frontend: \texttt{mobx} \\
                    \textit{Library} ini adalah manajemen \textit{state} yang digunakan untuk berbagai \textit{state}
                    antar komponen pada satu aplikasi. Aplikasi Oxam menggunakan \texttt{mobx} untuk menyimpan 
                    \textit{state} entitas yang akan digunakan antar halaman.
                \item Frontend: \texttt{mobx-react} \\
                    \textit{Library} \texttt{mobx-react} adalah \textit{binding} yang mengkoordinasi komponen
                    antar React dengan \textit{state management} \texttt{mobx}. Aplikasi Oxam menggunakan 
                    \textit{library} ini untuk membantu mengkoordinasikan komponen-komponen yang bergantung
                    pada \textit{state} yang disimpan oleh \texttt{mobx}.
                \item Frontend: \texttt{moment} \\
                    \textit{Library} ini digunakan untuk memanipulasi tanggal dan melakukan \textit{formatting}.
                    Aplikasi ini menggunakan \textit{library} ini karena \textit{library} \textit{react-timekeeper}
                    membutuhkan \texttt{moment}.
                \item Frontend: \texttt{opensans-npm-webfont} \\
                    \textit{Package} ini menyimpan berkas font yang digunakan pada web. Font OpenSans ini
                    digunakan pada aplikasi untuk memperjelas bentuk tulisan yang akan ditampilkan dilayar peserta
                    dan proyektor. OpenSans memiliki beberapa jenis ketebalan yang cocok digunakan pada konteks
                    tertentu.
                \item Frontend: \texttt{prop-types} \\
                    \texttt{prop-types} adalah \textit{library} yang digunakan oleh React untuk mevalidasi data 
                    properti yang diberikan oleh \textit{parent} komponen. Aplikasi ini menggunakan \textit{library}
                    \texttt{prop-types} untuk membuat beberapa komponen yang akan digunakan ulang pada beberapa
                    halaman.
                \item Frontend: \texttt{query-string} \\
                    \texttt{query-string} digunakan untuk \textit{parsing} bagian \textit{search} pada url dari
                    peramban. \textit{Library} ini akan memecah \textit{string} tersebut lalu mengubahnya menjadi
                    kelas yang dapat digunakan oleh pengembang. Aplikiasi ini menggunakan \textit{library} ini untuk
                    melakukan pengecekan \textit{return-path} pada beberapa halaman, terutama halaman \textit{editor}
                    entitas.
                \item Frontend: \texttt{react-countdown} \\
                    \texttt{react-countdown} digunakan untuk menampilan jam berbentuk \textit{countdown}. 
                    Aplikasi Oxam menggunakan \textit{libary} ini untuk menampilkan waktu yang tersisa
                    untuk ujian.
                \item Frontend: \texttt{react-dates} \\
                    \textit{Library} ini menyediakan komponen formuilir berbentuk pemilih tanggal dalam 
                    format kalender. Aplikasi Oxam menggunakan \textit{library} ini untuk membuat formulir
                    pemilih tanggal pada bidang bertipe tanggal.
                \item Frontend: \texttt{react-dropzone} \\
                    \texttt{react-dropzone} menyediakan \textit{binding} yang dapat digunakan untuk menangani
                    \textit{drag-drop} berkas pada suatu elemen HTML. Pada aplikasi ini, \textit{library}
                    \texttt{react-dropzone} digunakan untuk menangani \textit{event} peserta mengunggah berkas
                    dengan melakukan \textit{drag-drop} pada slot jawaban.
                \item Frontend: \texttt{react-if} \\
                    \texttt{react-if} menyediakan komponen percabangan, \textit{If-else}. Penggunaan komponen
                    tersebut akan meningkatkan kejelasan kode karena bentuk percabangan yang diberikan menjadi
                    lebih jelas. Aplikasi ini menggunakan \texttt{react-if} pada berbagai halaman untuk mengubah
                    beberapa komponen pada kasus tertentu.
                \item Frontend: \texttt{react-router-dom} \\
                    \textit{Library} ini digunakan untuk memetakan url tertentu pada frontend ke komponen tertentu.
                    Aplikasi menggunakan \textit{library} ini untuk memetakan alamat untuk login, lembar jawab ujian,
                    halaman admin dan dosen, serta sebagainya ke komponennya masing-masing.
                \item Frontend: \texttt{react-stepper-horizontal} \\
                    \textit{Library} ini adalah komponen UI yang digunakan untuk memvisualisasikan
                    jumlah langkah yang ada. Aplikasi menggunakan komponen ini untuk menampilkan jumlah 
                    langkah yang ada untuk membuat suatu ujian.
                \item Frontend: \texttt{react-timekeeper} \\
                    \textit{Libray} ini digunakan untuk menampilkan pemilih jam. Komponen UI ini akan menampilkan
                    jam dan pengguna dapat memilih angka yang terdapat pada tampilan tersebut untuk kemudian
                    dapat digunakan sebagai \textit{nput} jam. Aplikasi ini menggunakan \texttt{react-timekeeper}
                    untuk membantu mengisi bidang formulir tertentu.
                \item Frontend: \texttt{reactstrap} \\
                    \texttt{reactstrap} adalah \textit{binding} javascript untuk \texttt{bootstrap}. \textit{Library}
                    ini akan menggantikan javascript bawaan dari \texttt{bootstrap} yang menggunalan jQuery, menjadi
                    sepenuhnya menggunakan \textit{library} dari React. Aplikasi ini menggunakan \textit{library}
                    ini untuk menhidupkan fungsionalitas beberapa komponen UI yang disediakan oleh \texttt{bootstrap}.
                \item Frontend: \texttt{sass} \\
                    \textit{Library} \texttt{sass} digunakan untuk melakukan \textit{build} pada kode SASS menjadi CSS.
                    Aplikasi ini menggunakan \texttt{sass} untuk memperkecil hasil CSS yang dibuat, serta membuat
                    beberapa penggayaan khusus untuk komponen tertentu.
                \item Frontend: \texttt{suneditor-react} \\
                    \textit{Library} ini menyediakan \textit{binding} untuk React terhadap \textit{library}
                    \texttt{suneditor}. \texttt{suneditor} adalah WYSIWYG (\textit{What you see is what you get})
                    HTML Editor. Aplikasi ini menggunakan library ini untuk membuat badan notifikasi yang nantinya
                    akan disebar untuk setiap peserta.
            \end{itemize}
    \end{itemize}


\section{Hasil Implementasi}
    Pada bagian ini hasil implementasi sistem akan dijelaskan dengan bantuan tampilan antarmuka yang telah
    diimplentasi dalam React.
    
    \subsection{Halaman untuk Peserta}
        Halaman untuk peserta akan 
    \subsection{Halaman untuk Tim Admin}
    \subsection{Halaman untuk Dosen Pengawas / Layar Proyektor}
    \subsection{Halaman untuk Dosen Koordinator}
% antarmuka, satu satu

% bentuk foldernya

\section{Pengujian Experimental}
Pengujian experimental dilakukan dengan mendemokan pada tim Admin secara langsung. Pada pengujian
tersebut, Tim Admin akan membuat sebuah mata kuliah dan ujian, lalu mencoba fitur-fitur yang tersedia
pada sistem Oxam yang baru. Keberhasilan tersebut dilakukan dengan memberikan kuisioner yang kemudian
Tim Admin isi.

% todo more bahas.

\section{Pengujian Fungsional}
Pengujian fungsional dilakukan dengan bantuan \textit{Unit Testing}. Tes tersebut akan dilakukan
setiap kali kode diunggah ke repositori penyimpanan kode sumber dengan bantuan CI/CD. Tes tersebut
terdiri dari beberapa skenario dan langkah yang menentukan sebuah fitur dapat berjalan seperti
yang dikehendaki atau tidak. Tes-tes tersebut terdiri dari