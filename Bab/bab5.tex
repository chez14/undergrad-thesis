\chapter{Implementasi dan Pengujian}
\label{chap:implementasi-pengujian}

\section{Implementasi}

\subsection{Lingkungan Implementasi Perangkat Keras}

\subsection{Lingkungan Implementasi Perangkat Lunak}
Dengan bantuan Docker, lingkungan implementasi aplikasi dapat disederhanakan menjadi menentukan
kebutuhan \textit{runtime} aplikasi. Detil lingkungan aplikasi tersebut dapat dilihat pada:
\begin{itemize}
    \item Docker (version)\\
        Daftar \textit{image} yang digunakan untuk membangun \textit{container}:
        \begin{itemize}
            \item backend\\
                \texttt{php:7.3.8-apache}\\
                Berkas \texttt{Dockerfile} dapat dilihat pada lampiran %TODO: Lampiran
            
            \item database\\
                \texttt{mysql:5.7}\\
                Berkas \texttt{docker-compose.yml} dapat dilihat pada lampiran %TODO: Lampiran
        \end{itemize}
    \item Nodejs v12
\end{itemize}

\section{Hasil Implementasi}


\section{Pengujian Experimental}
Pengujian experimental dilakukan dengan mendemokan pada tim Admin secara langsung. Pada pengujian
tersebut, Tim Admin akan membuat sebuah mata kuliah dan ujian, lalu mencoba fitur-fitur yang tersedia
pada sistem Oxam yang baru. Keberhasilan tersebut dilakukan dengan memberikan kuisioner yang kemudian
Tim Admin isi.

% todo more bahas.

\section{Pengujian Fungsional}
Pengujian fungsional dilakukan dengan bantuan \textit{Unit Testing}. Tes tersebut akan dilakukan
setiap kali kode diunggah ke repositori penyimpanan kode sumber dengan bantuan CI/CD. Tes tersebut
terdiri dari beberapa skenario dan langkah yang menentukan sebuah fitur dapat berjalan seperti
yang dikehendaki atau tidak. Tes-tes tersebut terdiri dari