\chapter{Kesimpulan dan Saran}
\label{chap:kesimpulan-saran}

\section{Kesimpulan}
    Aplikasi pendukung ujian yang ada saat ini perlu dibuat ulang karena  beberapa masalah yang muncul, seperti
    dukungan untuk penanganan perpindahan tempat duduk peserta, penyebaran kredensial layanan, masalah waktu ujian
    yang tidak singkron, hingga terdapat NPM jenis yang baru. Aplikasi pendukung baru telah berhasil dibangun
    ulang untuk mengatasi hal-hal masalah tersebut.
    
    Aplikasi pendukung diimplementasi dengan teknologi aplikasi berbasis web yang dibantu dengan teknik REST API
    yang dapat dimanfaatkan untuk komunikasi antara sistem frontend dengan backend. Sistem frontend diimplementasi
    dengan bantuan \textit{library} Reactjs, dengan penggayaan yang dibantu dengan \texttt{bootstrap}. Sistem
    backend dibantu dengan \textit{framework} FatFree Framework dan \textit{library} pendukungnya. Sistem
    kemudian diimplementasikan diatas sistem Docker yang berjalan pada server lab. Selain itu, pada lingkungan
    pengembangan, terdapat sistem CI/CD yang berjalan diatas Gitlab CI yang digunakan untuk melakukan
    \textit{build} utnuk sistem frontend dan \textit{unit testing} pada sistem backend.
    
    Pengujian yang dilakukan pada aplikasi membuktikan  bahwa aplikasi dapat memenuhi kebutuhan baru yang muncul
    dari sistem yang saat ini berjalan. Hal ini didukung dengan kuisioner yang dilakukan pada Tim Administator
    yang berpengalaman dengan sistem yang saat ini berjalan, dengan hasil yang cukup memuaskan, 3.22 poin dari 4.
    
\section{Saran}
    Dibantu dengan kuisioner yang telah dilakukan pada saat pengujian, saran untuk penelitian selanjutnya adalah
    sebagai berikut:
    \begin{enumerate}
        \item Aplikasi dapat mendukung pengujian daring.
        \item Perbaikan terhadap beberapa tampilan antarmuka. \\
            Perbaikan tersebut dapat meliputi penggantian satuan untuk halaman konfirmasi pembuatan ujian dari jam
            ke menit, teks informasi dan perintah, serta ubah pemilihan peserta pada notifikasi menjadi tombol khusus.
            Selain itu mengganti beberapa judul bagian menjadi lebih jelas, seperti judul untuk \textit{editor}
            entri pada pengelola entitas.
        \item Sehabis waktu ujian habis, peserta seharusnya diberikan tampilan waktu ujian telah habis.
        \item Fitur notifikasi dibuat terintegrasi dengan peramban sehingga notifikasi muncul pada 
            sistem operasi juga.
        \item Fitur pengiriman email notifikasi seharusnya memunculkan informasi kapan email akan
            dikirim dan status pengirimannya.
        \item Memberikan informasi sisa notifikasi pengerjaan di taskbar Windows.
        \item Pada fitur perpindahan perserta, tombol pengunduhannya dibuat dapat diunduh ulang pada
            saat modal tertutup.
        \item Fitur perpindahan dan \textit{plotting} tempat duduk seharusnya menampilkan komputer
            yang sedang digunakan atau sedang ada masalah.
        \item Eksekusi \textit{script} yang otomatis, tidak perlu diunduh terlebih dahulu.
        \item Sistem memberikan peringatan jika terdapat peserta yang duduk bersebelahan.
    \end{enumerate}