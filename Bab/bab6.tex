\chapter{Kesimpulan dan Saran}
\label{chap:kesimpulan-saran}

\section{Kesimpulan}
    Aplikasi manajemen ujian yang digunakan saat ini, Oxam v4, perlu dibuat ulang karena beberapa masalah. 
    Aplikasi Oxam v5 telah berhasil dibangun ulang untuk mengatasi masalah-masalah yang ada.
    Aplikasi pendukung diimplementasi dengan teknologi aplikasi berbasis web yang dibantu dengan teknik REST API
    yang dapat dimanfaatkan untuk komunikasi antara sistem frontend dengan backend. Sistem frontend diimplementasi
    dengan bantuan \textit{library} Reactjs, dengan penggayaan yang dibantu dengan \texttt{bootstrap}. Sistem
    backend dibantu dengan \textit{framework} FatFree Framework dan \textit{library} pendukungnya. Sistem
    kemudian diimplementasikan diatas sistem Docker yang berjalan pada server lab. Selain itu, pada lingkungan
    pengembangan, terdapat sistem CI/CD yang berjalan diatas Gitlab CI yang digunakan untuk melakukan
    \textit{build} utnuk sistem frontend dan \textit{unit testing} pada sistem backend.
    
    Pengujian yang dilakukan pada aplikasi membuktikan  bahwa aplikasi dapat memenuhi kebutuhan baru yang muncul
    dari sistem yang saat ini berjalan. Hal ini didukung dengan kuisioner yang dilakukan pada tim administator
    yang berpengalaman dengan sistem yang saat ini berjalan, dengan hasil yang cukup memuaskan, 3.22 poin dari 4.
    
\section{Saran}
    Saran untuk penelitian selanjutnya adalah sebagai berikut:
    \begin{enumerate}
        \item Sistem dibuat menjadi lebih stabil.
        \item Ditambahkan sebuah sistem yang dapat melakukan pembaharuan \textit{dependency library} yang
            digunakan oleh Oxam v5 secara otomatis dan terintegrasi.
        \item Meminta pendapat dari tim admin untuk pengalaman pengembangan aplikasi.
        \item Membuat dokumentasi aplikasi dan proyek yang dapat dilihat untuk publik dan kontributor.
    \end{enumerate}