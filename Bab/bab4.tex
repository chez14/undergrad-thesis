\chapter{Perancangan}
\label{chap:perancangan}

\section{Rancangan Antarmuka}
    Peracangan antarmuka dilakukan dengan membuat \textit{mockup} aplikasi. Desain \textit{mockup} dibagi 
    menjadi tiga bagian besar berdasarkan peranan penggunanya.
    Kemudian tampilan antarmuka diimplementasi dengan bantuan Bootstrap dan \texttt{reactstrap} untuk
    membantu pengembangan aplikasi. Pembahasan akan dimulai dengan rancangan 
    antarmuka untuk Perserta, lalu perancangan antarmuka untuk Admin, Dosen Pengawas, dan kemudian rancangan 
    antarmuka tambahan.

\subsection{Rancangan Antarmuka untuk Peserta}
    Rancangan antarmuka untuk peserta didasari dari kasus penggunaan dari aplikasi. Berdasarkan hasil analisis
    pada survei, peserta sudah merasa nyaman dengan antarmuka pada sistem yang lama. Sehingga penulis
    mengadopsi antarmuka yang lama dan diperbarui dengan kebutuhan yang baru. Beberapa kebutuhan yang baru
    seperti penutupan dan pembukaan lembar jawaban, informasi ujian dan waktu, serta fitur notifikasi untuk
    kredensial login dan pengumuman lainnya.
    
     % Screen ujian kosong
    \begin{figure}[H]
        \centering
        \includegraphics[width=0.7\paperwidth]{Gambar/mockups/Mockup--Peserta - Blankstate.pdf}
        \caption{Tampilan rancangan antarmuka untuk peserta, saat ujian sedang tidak berjalan.}
        \label{fig:mockup_peserta_blankstate}
    \end{figure}
     Tampilan antarmuka pertama yang akan dibahas adalah tampilan untuk menunjukkan informasi bahwa
     tidak ada ujian yang sedang berjalan, dapat dilihat pada Gambar \ref{fig:mockup_peserta_blankstate}. 
     Hal ini diperuntukkan untuk membantu tim admin melakukan pengecekan lebih cepat karena 
     tampilan layar yang beda secara
     signifikan. Selain itu, tampilan pesan yang cukup ramah untuk pengguna menampilkan 
     informasi bahwa mungkin peserta mengambil tempat duduk yang salah.

    % Screen ujian AKAN mulai
    \begin{figure}[H]
        \centering
        \includegraphics[width=0.7\paperwidth]{Gambar/mockups/Mockup--Peserta - Prestartstate.pdf}
        \caption{Tampilan rancangan antarmuka untuk peserta, saat ujian akan berjalan.}
        \label{fig:mockup_peserta_prestartstate}
    \end{figure}
    Rancangan berikutnya adalah tampilan antarmuka untuk memulai ujian, dapat dilihat pada 
    \ref{fig:mockup_peserta_prestartstate}. Tampilan ini akan ditampilkan
    pada saat ujian akan dimulai di ruangan. Tampilan ini rencananya akan dibuat untuk menahan
    perserta untuk langsung mengakses lembar jawaban ujian. Pada bagian ini terdapat beberapa informasi
    tentang peserta (poin 1), fitur notifikasi untuk mengecek kredensial login lainnya (poin 2),
    dan informasi lengkap ujian dalam bentuk tabel (poin 3).
    Selain itu terdapat deskripsi kecil untuk membantu peserta mengerti konteks dari tampilan
    layar ini. 
    
    % Screen ujian buka lembar jawab
    \begin{figure}
        \centering
        \includegraphics[width=0.7\paperwidth]{Gambar/mockups/Mockup--Peserta - Startstate.pdf}
        \caption{Tampilan rancangan antarmuka untuk peserta, saat ujian sedang berjalan.}
        \label{fig:mockup_peserta_activestate}
    \end{figure}
    Antarmuka berikutnya adalah antarmuka saat ujian sedang berjalan, dapat dilihat pada Gambar 
    \ref{fig:mockup_peserta_activestate}. Tampilan rancangan antarmuka ini memiliki beberapa bagian
    yang mirip dengan tampilan sebelumnya seperti bagian kepala (poin 3), namun dilengkapi dengan
    informasi singkat tentang ujian (poin 1) dan timer ujian serta lokasi tempat duduk (poin 2).
    Selain itu, untuk mempercepat admin melihat status pada layar, bagian kepala diwarnai berbeda
    dengan tampilan sebelumnya (dari kuning menjadi biru).
    
    Bagian yang ditunjukkan oleh poin 4 adalah bagian pengunggahan. Bagian ini memiliki informasi
    tentang nama berkas yang harus dikumpulkan dan tombol unggah dan unduh. Tombol unduh akan dapat
    diklik pada saat peserta berhasil mengunggah. Tombol unduh ditambahkan untuk alasan pengecekkan.
    Berdasarkan survei lapangan, peserta biasanya akan diarahkan untuk melakukan pengecekkan ulang.
    Pengencekan tersebut melibatkan peserta untuk mengunduh berkas ujian untuk kemudian dipastikan
    apakah berkas jawaban tersebut sudah sesuai dengan apa yang mereka kirimkan.
    
    % Screen ujian notifikasi
    \begin{figure}
        \centering
        \includegraphics[width=0.7\paperwidth]{Gambar/mockups/Mockup--Peserta - Notif.pdf}
        \caption{Tampilan rancangan antarmuka keseluruhan untuk notifikasi peserta. \\
            (A) Daftar notifikasi; (B) Notifikasi terbuka.}
        \label{fig:mockup_peserta_notif}
    \end{figure}
    Rancangan berikutnya adalah fitur notifikasi. Fitur ini akan dimuncuklkan sejak tahap ujian sedang akan
    dimulai. Pada Gambar \ref{fig:mockup_peserta_prestartstate} dan \ref{fig:mockup_peserta_activestate}
    dapat dilihat bahwa terdapat tombol dengan lambang 'N' dekat dengan informasi singkat peserta.
    tombol tersebut adalah tombol notifikasi. Jika tombol tersebut diklik, maka akan tampil sebuah
    daftar notifikasi dalam bentuk \textit{dropdown} (Perhatikan Gambar \ref{fig:mockup_peserta_notif},
    bagian A). Pada rancangan tampilan, peserta dapat melihat beberapa notifikasi yang telah
    diberikan untuk peserta tersebut (perhatikan Poin 1). Jika Poin 1 diklik, maka notifikasi akan
    ditampilkan secara modal (Perhatikan Gambar \ref{fig:mockup_peserta_notif}, bagian B).
    Tampilan interaktif dengan modal dipilih untuk menunjukan bahwa peserta dapat menutup
    modal tersebut tanpa kehilangan formulir lembar pengumpulan jawaban ujian.
    Selain itu dengan menggunakan modal, tataletak yang sudah ada tidak berubah, dan memperkecil
    \textit{learning curve} untuk peserta membiasakan diri dengan aplikasi ini.
    
\subsection{Rancangan Antarmuka untuk Admin}
    
    % Screen admin login
\subsubsection{Otentikasi}
    \begin{figure}
        \centering
        \includegraphics[width=0.7\paperwidth]{Gambar/mockups/Mockup--Admin - Login.pdf}
        \caption{Tampilan rancangan antarmuka untuk halaman otentikasi untuk Admin}
        \label{fig:mockup_admin_login}
    \end{figure}
    Untuk dapat mengakses halaman admin panel, pengguna diharuskan untuk melakukan otentikasi
    terlebih dahulu. Rancangan tampilan admin panel tersebut dapat dilihat pada Gambar 
    \ref{fig:mockup_admin_login}. Pada rancangan tampilan, terdapat beberapa bagian yang dapat
    pengguna isi untuk login. Bagian yang ditunjukan oleh poin 1 adalah bidang username dari
    admin. Bidang ini direncanakan untuk dapat menerima namapengguna dan email. Lalu bagian
    berikutnya adalah bidang yang ditunjukkan oleh poin 2. Bidang tersebut adalah bidang kata sandi.
    Lalu pada bagian yang ditunjukan oleh poin 3 dan 4 adalah tombol aksi untuk melakukan login.
    Tombol yang ditunjukkan oleh poin 3 tidak memanfaatkan bidang namapengguna dan kata sandi,
    namun akan langsung melakukan pemanggilan API ke backend untuk melakukan otentikasi
    secara nir-kata sandi dengan bantuan IP. Proses ini akan dijelaskan lebih lanjut pada bagian
    rancangan antarmuka untuk dosen pengawas. Bagian yang ditunjukkan pada poin 4 adalah tombol
    login sesungguhnya. Pengguna yang telah dipastikan sebagai admin akan dapat melakukan login
    dan diarahkan menuju daftar ujian. Namun jika pengguna memasukkan kredensial login yang salah
    sebuah pesan kesalahan akan dimunculkan dalam bentuk \textit{Alert}.
    
\subsubsection{Daftar Ujian}
    % Screen admin exam listing
    \begin{figure}
        \centering
        \includegraphics{Gambar/mockups/Mockup--Admin - Exam Listing.pdf}
        \caption{Rancangan antarmuka untuk daftar ujian pada halaman Admin Panel.}
        \label{fig:mockup_admin_exam_listing}
    \end{figure}
    Rancangan antarmuka untuk manajemen ujian dimulai dengan halaman daftar ujian, dapat dilihat pada Gambar
    \ref{fig:mockup_admin_exam_listing}. Pada rancangan tersebut secara garis besar menampilkan informasi tentang
    ujian-ujian yang sudah terdaftar pada sistem berserta informasi singkatnya. Bagian-bagian yang ditunjukkan
    dengan poin dijelaskan sebagai berikut:
    \begin{itemize}
        \item Bagian yang ditunjukkan pada Poin 1 adalah daftar ujian yang sudah teregistrasi pada sistem.
            Daftar tersebut disajikan dalam bentuk tabel (poin 1) yang kemudian dapat diklik untuk menuju halaman detil ujian.
            Tabel akan menunjukan setidaknya informasi mata kuliah, periode ujian, UTS atau UAS, \textit{shift},
            serta tanggal dan jam mulai serta berakhirnya ujian jika ujian sudah dimulai.
        
        \item Bagian yang ditunjukkan pada Poin 2 adalah tombol aksi untuk membuat ujian yang baru.
    \end{itemize}
    
\subsubsection{Pembuatan Ujian Baru}
    % Screen admin exam create (3-4 screens)
    \begin{figure}
        \centering
        \includegraphics{Gambar/mockups/Mockup--Admin - NewExam-Step 1.pdf}
        \caption{Rancangan antarmuka untuk membuat ujian baru, langkah pertama dari empat.}
        \label{fig:mockup_admin_exam_create-1}
    \end{figure}
    Untuk mendaftarkan ujian Admin harus mengisi formulir yang disediakan oleh sistem. Formulir akan
    dipisah menjadi beberapa langkah untuk mempermudah pengecekan. Fomulir pada langkah pertama dapat
    dilihat pada Gambar \ref{fig:mockup_admin_exam_create-1}. Halaman ini dapat diakses pada saat Admin
    menekan tombol yang ditujukkan oleh poin 2 pada gambar \ref{fig:mockup_admin_exam_listing}.
    Poin-poin yang terdapat pada gambar dijelaskan sebagai berikut:
    \begin{itemize}
        \item Poin 1 adalah tampilan \textit{stepper}. \textit{Stepper} menunjukan langkah yang harus ditempuh oleh
            tim admin untuk membuat mengisi formulir ini. Untuk saat ini terdapat empat langkah besar yang terdiri
            dari:
                \begin{enumerate}
                    \item Pengisian detil ujian (Exam Details).
                    \item Pangalokasian tempat duduk ujian (Seat Plotting).
                    \item Konfirmasi (Confirmation).
                    \item Penyelesaian (Finish).
                \end{enumerate}
            Seperti yang telah dideskripsikan, halaman formulir ini berada pada tahap pengisian detil ujian.
            
        \item Poin 2 menunjukkan bagian detil ujian seperti tipe ujian yang dapat diisi sebagai UTS ataupun UAS,
            mata kuliah dati ujian ini, shift, mulai pada tanggaal jam berapa, hingga durasi waktu ujian.
            Informasi yang ditampilkan pada bagian ini berhubungan langusng dengan entitas \texttt{Exam}. Sesuai dengan
            jenis bidangnya, jika sebuah bidang memiliki pilihan spesifik, maka bidang tersebut akan diimplementasi
            dengan \textit{dropdown}. Sebagai contoh bidang tipe ujian hanya akan memiliki pilihan UTS ataupun UAS,
            maka bidang diimplementasi dengan \textit{dropdown} dengan nilai pilihan yang sudah disebutkan.
            Untuk bidang tanggal dan jam, implementasi akan dilakukan dengan bantuan dari \textit{date picker}.
            
        \item Poin 3 menunjukkan daftar peserta yang akan mengikuti ujian ini. Daftar peserta berisi daftar NPM
            yang mengikuti ujian, dipisah dengan sebuah enter.
            
        \item Poin 4 adalah sebuah tombol unggah yang akan diproses langsung pada browser untuk memasukkan daftar
            peserta. Tombol ini ditambahkan dengan alasan kompabilitas UX dengan sistem yang lama.
            
        \item Poin 5 adalah sebuah tombol aksi untuk menuju langkah berikutnya, pengalokasian tempat duduk ujian.
        
    \end{itemize}
    
    
    \begin{figure}
        \centering
        \includegraphics{Gambar/mockups/Mockup--Admin - NewExam-Step 2.pdf}
        \caption{Rancangan antarmuka untuk membuat ujian baru, langkah kedua dari empat.}
        \label{fig:mockup_admin_exam_create-2}
    \end{figure}
    
    Setelah tim admin selesai mengisi informasi detil ujian, maka langkah selanjutnya yang akan dilakukan adalah 
    mengalokasi tempat duduk ujian. Alokasi ini dilakukan dengan bantuan peta ruangan yang telah didefinisikan
    sebelumnya oleh sistem. Rancangan antarmuka tersebut dapat dilihat pada Gambar \ref{fig:mockup_admin_exam_create-2}.
    Secara garis besar tampilan memiliki tataletak yang mirip dengan langkah sebelumnya: Judul langkah diatas, diikuti
    dengan \textit{stepper} di bawahnya. Namun bagian yang ditunjukkan pada poin 1 akan berisi peta tempat duduk
    ruangan. Peta tempat duduk ini nantinya akan memiki tampilan yang sama dengan yang ditampilkan pada proyektor.
    
    \begin{figure}
        \centering
        \includegraphics{Gambar/mockups/Mockup--Admin - NewExam-Step 3.pdf}
        \caption{Rancangan antarmuka untuk membuat ujian baru, langkah ketiga dari empat.}
        \label{fig:mockup_admin_exam_create-3}
    \end{figure}
    Setelah tempat duduk dialokasi, langkah berikutnya adalah konfirmasi. Tim admin diharapkan untuk
    mengkonfirmasi detil ujian yang akan dibuat sebelum akhirnya difinialisasi oleh sistem. Rancangan tampilan
    layar tersebut dapat diperhatikan pada \ref{fig:mockup_admin_exam_create-3}. Pada rancangan tampilan
    tersebut terdapat informasi singkat dari ujian yang akan dibuat. Bagian yang ditunjukkan dengan poin-poin
    pada gambar akan dijelaskan sebagai berikut:
    
    \begin{itemize}
        \item Poin 1 menunjukkan informasi tentang nama matakuliah, banyak peserta, informasi shift serta informasi
            jadwal ujian tersebut.
            
        \item Poin 2 adalah tombol aksi untuk melakukan pembuatan ujian. Tombol ini akan melakukan beberapa pemanggilan
            ke sistem backend sekaligus sebelum akhirnya memindahkan tim admin ke tahap berikutnya.
    \end{itemize}
    
    
    \begin{figure}
        \centering
        \includegraphics{Gambar/mockups/Mockup--Admin - NewExam-Step 4.pdf}
        \caption{Rancangan antarmuka untuk membuat ujian baru, langkah keempat dari empat.}
        \label{fig:mockup_admin_exam_create-4}
    \end{figure}
    Antarmuka terkahir untuk tahap pembuatan ujian adalah tahap penyelesaian. Tampilan antarmuka ini dapat dilihat
    pada gambar \ref{fig:mockup_admin_exam_create-4}. Tampilan ini berfungsi untuk menampilkan respon dari sistem
    bahwa ujian telah berhasil dibuat dengan informasi yang telah diberikan. Antarmuka pada tahap ini hanya akan
    berisi pesan respon singkat disertai dengan tombol "Lihat Ujian" tombol ini akan mengarahkan tim admin ke
    halaman ujian.
    
    % Screen admin exam details (mayan banyak)
\subsubsection{Detil Ujian}
    \begin{figure}
        \centering
        \includegraphics[width=0.75\paperwidth]{Gambar/mockups/Mockup--Admin - Exam Details.pdf}
        \caption{Rancangan antarmuka untuk tampilan detil ujian.}
        \label{fig:mockup_admin_exam_details}
    \end{figure}
    
    \begin{figure}
        \centering
        \includegraphics[width=0.75\paperwidth]{Gambar/mockups/Mockup--Admin - Slot Jawaban.pdf}
        \caption{Rancangan antarmuka untuk modal slot jawaban baru.}
        \label{fig:mockup_admin_exam_det_answer_slot}
    \end{figure}
    
    \begin{figure}
        \centering
        \includegraphics[width=0.75\paperwidth]{Gambar/mockups/Mockup--Admin - Tambah Autoreport.pdf}
        \caption{Rancangan antarmuka untuk modal pelaporan otomatis baru.}
        \label{fig:mockup_admin_exam_det_autoreport}
    \end{figure}
    
    Halaman detil ujian akan memuat seluruh informasi ujian yang ditampung pada entitas \texttt{Exam} dan
    beberapa entitas lainnya yang terhubung dengan entitas ini. Rancangan tampilan detil ujian dapat
    dilihat pada gambar \ref{fig:mockup_admin_exam_details}. Secara garis besar, tampilan memiki beberapa
    bagian yang spesifik dengan perannya. Hal ini dibuat demikian untuk mempermudah tim admin memindai
    informasi pada saat membuka beberapa \textit{tab} pada peramban. Bagian-bagian yang ditunjukan dengan
    poin bewarna kuning akan dijelaskan sebagai berikut:
    \begin{itemize}
        \item Poin 1 adalah bagian informasi singkat tentang ujian. Informasi akan berisi minimal
            kode mata kuliah, shift (jika ada), dan nama matakuliah.
            
        \item Poin 2 akan berisi status dari ujian tersebut, serta sisa waktu yang ada jika ujian tersebut 
            sedang berjalan.
            
        \item Poin 3 adalah kotak alat untuk ujian ini. Kotak alat tersebut berisi:
            \begin{itemize}
                \item Tombol print daftar hadir.
                \item Tombol layar proyektor.
                \item Tombol unduh script untuk membuat folder dan berkas ujian.
                \item Tombol pemindahan peserta.
            \end{itemize}
            Kotak alat diimplementasikan untuk memenuhi kebutuhan yang muncul dari admin sebelumnya.
            
        \item Poin 4 menunjukkan informasi ujian dengan lebih detil, disajikan dalam bentuk tabel.
        
        \item Poin 5 adalah bagian pengelolaan Slot Jawaban. Bagian ini terdiri dari beberapa tombol.
            \begin{itemize}
                \item Pada bagian atas, terdapat tombol Zip yang digunakan untuk melakukan pengumpulan berkas
                    jawaban menjadi sebuah \textit{archive} zip. Peramban akan kemudian mengunduh berkas tersebut
                    dan tim admin dapat mengirimkan berkas jawaban tersebut secara manual ke dosen.
                    
                \item Pada sebelah kanan tombol Zip, terdapat tombol tambah yang akan membuka sebuah
                    modal yang akan menanyakan format slot jawaban yang ada. Modal tersebut dapat dilihat
                    pada Gambar \ref{fig:mockup_admin_exam_det_answer_slot}. Pada poin 1 dari Gambar
                    \ref{fig:mockup_admin_exam_det_answer_slot}, format yang diterima dapat berupa
                    format \textit{xxyyy} seperti pada sistem sebelummnya.
                    
                \item Untuk setiap slot jawaban yang ada, akan terdapat tombol ubah untuk mengubah lembar
                    jawab, dan tombol hapus yang dapat digunakan untuk menghapus entri tersebut.
            \end{itemize}
            
        \item Poin 6 menunjukan bagian fitur notifikasi yang akan ditampilkan pada peserta ujian.
            Pada tampilan ini, terdapat beberapa tombol yang dapat digunakan untuk memanipulasi ujian
            tersebut. Tombol tombol tersebut adalah
                \begin{itemize}
                    \item Tombol Tambah untuk membuat notifikasi baru. Pada saat diklik, sistem akan
                        menampilkan sebuah modal yang menanyakan informasi jenis notifikasi. 
                        Karena alur yang cukup panjang, alur kerja untuk pembuatan notifikasi akan 
                        dibahas setelah bagian ini.
                        
                    \item Untuk setiap entri notifikasi akan memiliki dua buah tombol yang dapat digunakan untuk
                        mengubah dan menghapus entri tersebut.
                \end{itemize}
            
        \item Poin 7 menunjukan bagian fitur pelaporan otomatis. Fitur pelaporan otomatis ini memiliki beberapa 
            tombol aksi seperti:
           \begin{itemize}
               \item Tombol Tambah yang dapat digunakan untuk menambahkan autoreport baru. Jika tombol ini
                    diklik, sistem akan memunculkan modal yang menanyakan pada siapa report ini akan dikirimkan.
                    rancangan modal tersebut dapat dilihat pada Gambar \ref{fig:mockup_admin_exam_det_autoreport}.
                    Bagian yang ditunjukan pada poin 1 dari rancangan tersebut dapat diisi dengan daftar email
                    yang dipisah dengan tanda koma.
                    
                \item Untuk setiap entri pada bagian ini memiliki dua tombol yang dapat digunakan untuk
                    mengirimkan email, dan penghapusan entri.
           \end{itemize}
    \end{itemize}
    
\subsubsection{Detil Ujian: Fitur notifikasi}
    \begin{sidewaysfigure}
        \centering
        \includegraphics{Gambar/mockups/Mockup--Admin - Notif.pdf}
        \caption{Rancangan antarmuka untuk beberapa tampilan modal jenis notifikasi. \\
            (A) Modal jenis notifikasi; (B) Notifikasi Kata Sandi; (C) Notifikasi Lainnya.}
        \label{fig:mockup_admin_exam_det_notif}
    \end{sidewaysfigure}
    
    Pada bagian notifikasi, rancangan tampilan antarmuka memiliki beberapa bagian tergantung dari
    jenis notifikasi yang ingin disebarkan ke setiap peserta. Seperti yang dapat dilihat pada
    \ref{fig:mockup_admin_exam_det_notif}, rancangan tampilan antarmuka terdapat tiga bagian.
    
    Bagian yang ditunjukan dengan label A adalah modal pertama yang muncul pada saat tim admin
    menekan tombol tambah. Modal akan memiliki dua buah tombol yang merepresentasikan jenis
    yang didukung oleh fitur ini: Kata sandi (poin 1) dan lainnya (poin 2). Jika tombol untuk jenis kata sandi
    dipilih, maka tim admin akan disajikan modal dengan label B. Sebaliknya, jika tombol untuk jenis lainnya
    dipilih, maka sistem akan menyajikan modal dengan label C.
    
    Modal yang ditunjukkan dengan label B adalah modal untuk mengisi informasi layanan dan daftar kata sandi
    yang akan disebarkan. Bagian-bagian yang terdapat pada modal ini adalah sebagai berikut
    \begin{itemize}
        \item Poin 3 menunjukan informasi tentang layanan dari kata sandi yang akan disebar.
        
        \item Poin 4 adalah daftar kata sandi dalam bentuk teks yang berisi informasi namapengguna
            dan kata sandi, serta npm yang dipisah dengan karakter enter untuk setiap entrinya.
            
        \item Poin 5 adalah tabel daftar peserta yang akan menerima kredensial nama pengguna dan kata sandi
            dari tim admin. Tabel ini ditambahkan untuk tim admin melakukan pengecekan sebelum notifikasi
            disebar.
    \end{itemize}
    
    Kemudian, modal yang ditunjukkan dengan label C adalah modal untuk jenis notifikasi lainnya. Notifikasi ini
    nantinya akan dapat ditargetkan untuk peserta-peserta tertentu. Bagian-bagian yang terdapat pada modal
    ini dijelaskan sebagai berikut
    \begin{itemize}
        \item Poin 6 menunjukan bidang judul yang akan disampaikan ke peserta ujian.
        
        \item Poin 7 menunjukan bidang yang digunakan untuk mespesifikasikan penerima notifikasi ujian.
        
        \item Poin 8 adalah isi dari notifikasi tersebut.
    \end{itemize}
    
    % Absensi (pintu dan tanda tangan)
\subsubsection{Detil Ujian: Daftar Hadir}
    \begin{figure}
        \centering
        \includegraphics{Gambar/mockups/Mockup--Admin - Daftar Hadir.pdf}
        \caption{Rancangan tampilan untuk daftar hadir peserta ujian.}
        \label{fig:admin_mockup_exam_det_daftar-hadir}
    \end{figure}
    
    Salah satu bagian dari kotak alat yang disediakan untuk tim admin adalah daftar hadir. Rancangan tampilan
    daftar harid dapat dilihat pada Gambar \ref{fig:admin_mockup_exam_det_daftar-hadir}. Daftar hadir tersebut
    harus dapat dicetak dengan pencentak. Daftar hadir ini akan terdiri dari sebuah halaman yang berisi beberapa
    bagian penting, seperti
    \begin{itemize}
        \item Poin 1, informasi singkat ujian seperti informasi UTS/UAS, kode mata kuliah, nama mata kuliah,
            shift (jika ada), tanggal mulai, dan ruangan.
        
        \item Poin 2, daftar peserta dengan pengurutan kolom tertentu. Jika daftar hadir ditempelkan
            pada pintu, maka daftar hadir harus diurutkan berdasarkan NPMnya. Jika daftar
            hadir digunakan untuk absensi, maka daftar tersebut harus diurutkan berdasarkan nomor komputernya.
            
        \item Poin 3, daftar hadir pada ruangan lain yang setiap daftar hadirnya terdiri dari poin 1 dan 2.
            Daftar hadir ini harus dapat terpisah menjadi halaman baru pada saat dicetak.
    \end{itemize}
    
    Tampilan daftar hadir ini harus dapat dicetak oleh pencetak atau \textit{printer}. Sehingga setiap halaman
    harus memiliki logo UNPAR dan informasi fakultas.
    
    % Timer
\subsubsection{Detil Ujian: Layar Proyektor}
    \begin{figure}
        \centering
        \includegraphics{Gambar/mockups/Mockup--Admin - Layar Proyektor.pdf}
        \caption{Rancangan antarmuka untuk tampilan pada layar proyektor.}
        \label{fig:mockup_admin_det_projector}
    \end{figure}
    
    % TODO: Ganti wording.
    Fitur berikutnya adalah tampilan timer untuk proyektor. Rancangan tampilan dapat dilihat pada
    Gambar \ref{fig:mockup_admin_det_projector}. Karena akun admin tidak memiliki relasi dengan 
    lokasi ruangan, maka tampilan ini ditambahkan sebagai cadangan dan digunakan hanya pada saat darurat.
    Tampilan ini akan memiliki informasi singkat tentang ujian yang akan diadakan dan durasinya (poin 1),
    sisa waktu (poin 2) dan tombol kontrol timer (poin 3).
    Tombol kontrol timer ini akan terhubung dengan sistem untuk membuka dan menutup lembar jawab.
    

    % Pindah
\subsubsection{Detil Ujian: Pindah Peserta}
    \begin{figure}
        \centering
        \includegraphics[height=0.7\paperheight]{Gambar/mockups/Mockup--Admin - Pindah Peserta.pdf}
        \caption{Rancangan tampilan untuk pemindahan peserta. \\
            (A) Daftar pemindahan; (B) Pencari peserta target; (C) Pencari komputer target; (D) Konfirmasi dan
            pengunduhan \textit{script} pemindahan.}
        \label{fig:mockup_admin_exam_det_migrator}
    \end{figure}
    
    Fitur pada kotak alat berikutnya adalah fitur untuk memindahkan peserta. Rancangan tampilan pada fitur ini
    dapat dilihat pada Gambar \ref{fig:mockup_admin_exam_det_migrator}. Fitur pemindahan ini akan melalui tahapan
    seperti memilih peserta yang akan dipindahkan (Bagian B), memilih komputer tujuan (Bagian C), lalu
    eksekusi script yang diberikan oleh sistem (Bagian D).
    Poin-poin yang ditujukkan pada sistem akan dijelaskan sebagai berikut:
    \begin{itemize}
        \item Modal Migrator (Bagian A)
            \begin{itemize}
                \item Poin 1 adalah tombol tambah yang akan membuka Modal yang terdapat pada bagian B.
                
                \item Poin 2 menunjukkan daftar peserta yang akan dipindahkan beserta komputer tujuannya.
                
                \item Poin 3 adalah tombol eksekusi pemindahan pada level basis data dan bangkitkan sebuah
                    \textit{script} untuk melakukan pemindahan lembar kerja peserta.
            \end{itemize}
        
        \item Modal Tambah Peserta (Bagian B)
            \begin{itemize}
                \item Poin 4 adalah bidang pencarian cepat untuk membantu tim admin mencari peserta yang 
                    ingin dipindah.
                    
                \item Poin 5 menunjukkan daftar peserta yang terdapat pada ujian ini, dengan sebuah tombol
                    pilih pada tiap entri untuk memilih peserta bersangkutan untuk melakukan pemindahan.
            \end{itemize}
            
        \item Modal Pilih Komputer (Bagian C)
            \item Poin 6 menunjukkan informasi peserta mana yang akan dipindah. Bagian ini ditambahkan
                untuk menyakinkan admin bahwa admin telah memilih peserta yang benar.
                
            \item Poin 7 adalah peta tempat duduk ujian pada ruangan tertentu. Komputer pada peta tersebut dapat
                diklik untuk dipilih. Saat tempat duduk selesai dipilih, maka modal akan kembali ke Bagian A,
                hingga seluruh peserta yang ingin dipindah telah didaftarkan seluruhnya.
    \end{itemize}
    
    % Tampilan mobile (minipanel)
\subsubsection{Detil Ujian: Minipanel}
    \begin{figure}
        \centering
        \includegraphics{Gambar/mockups/Mockup--Admin - Minipanel.pdf}
        \caption{Rancangan antarmuka untuk Minipanel}
        \label{fig:mockup_admin_minipanel}
    \end{figure}
    Tampilan minipanel adalah tampilan untuk perangkat bergerak yang dapat digunakan oleh tim admin
    untuk memicu mulainya timer ujian. Pemicu tersebut akan terhubung langsung dengan sistem back-end
    dan membuka lembar jawab untuk peserta. Rancangan tampilan dapat dilihat pada Gambar
    \ref{fig:mockup_admin_minipanel}. Secara garis besar, rangcangan tampilan memiliki informasi
    singkat ujian pada atas halaman, satu set timer dengan kontrolnya, serta tabel detil ujian
    pada bagian bawah halaman.
    
    Bagian-bagian yang ditunjukan oleh poin dideskripsikan sebagai berikut:
    \begin{itemize}
        \item Poin 1 mengandung informasi singkat tentang kode mata kuliah yang dipilih.
        
        \item Poin 2 adalah bagian satu set timer dengan kontrolnya. Tombol dibuat lebih besar
            agar lebih mudah untuk ditekan. Selain itu kontrol juga akan tersedia pada bagian bawah
            halaman untuk mempermudah pemicu pada perangkat bergerak.
            
        \item Poin 3 adalah informasi detil ujian dalam bentuk tabel.
    \end{itemize}
    
    % Screen admin entity edit
\subsubsection{Pengelola Entitas: Daftar Entri}
    \begin{figure}
        \centering
        \includegraphics{Gambar/mockups/Mockup--Admin - Entity Lister.pdf}
        \caption{Rancangan antarmuka untuk daftar entri entitas}
        \label{fig:mockup_admin_entity_lister}
    \end{figure}
    Selain ujian, tim admin dapat mengelola entri dari entitas. Rancangan antarmuka pertama untuk 
    pengelola entri adalah daftar entri. Setiap entitas yang ada akan memiliki sebuah daftar yang 
    disajikan dengan tabel terdefinisi. Rancagan antarmuka tersebut dapat dilihat pada Gambar
    \ref{fig:mockup_admin_entity_lister}.
    
    Seperti rancangan antarmuka lainnya, bagian yang ditunjukkan oleh poin-poin dideskripsikan sebagai berikut:
    \begin{itemize}
        \item Poin 1 adalah tombol untuk membuat entri baru. Tombol ini akan mengarahkan admin
            ke halaman \textit{editor} dengan bidang yang tidak terisi.
            
        \item Poin 2 adalah daftar entri yang terdapat pada entitas tersebut. Setiap entri akan
            memiliki tombol ubah dan hapus. Tombol ubah kan mengarahkan admin ke halaman \textit{editor}
            dengan bidang yang terisi dari entri terpilih. Tombol hapus akan memunculkan modal konfirmasi
            penghapusan entri.
            
            Daftar entri disajikan dalam bentuk tabel yang telah didefinisikan konfigurasinya. Tiap entri
            yang dapat dikelola akan memiliki sebuah daftar definisi yang ditentukan oleh pengembang. Berkas
            definisi tersebut akan dapat menentukan aturan kolom-kolom tertentu yang ingin ditampilkan.
    \end{itemize}
    
\subsubsection{Pengelola Entitas: Buat dan Ubah Entri Baru}
    \begin{figure}
        \centering
        \includegraphics[width=0.75\paperwidth]{Gambar/mockups/Mockup--Admin - Entity Editor.pdf}
        \caption{Rancangan antarmuka untuk \textit{editor} entri.\\ (A) Buat baru; (B) Ubah yang sudah ada.}
        \label{fig:mockup_admin_entity_editor}
    \end{figure}
    Entri dari entitas dapat diubah dengan bantuan \textit{editor}. Rancangan tampilan \textit{editor} tersebut
    dapat dilihat pada Gambar \ref{fig:mockup_admin_entity_editor}. Secara garis besar, tampilan ini akan
    memiliki tanggung jawab untuk membuat dan mengubah data entri baru maupun yang sudah ada. Bagian-bagian yang
    ditunjukan dengan poin dideskripsikan sebagai berikut
    \begin{itemize}
        \item Poin 1a judul menunjukkan informasi status bahwa \textit{editor} akan melakukan perubahan
            pada entitas dengan id 14. Jika status editor adalah untuk membuat entri baru, maka judul akan
            tampil mirip seperti poin 1b.
        
        \item Poin 2 adalah daftar atribut yang dapat diubah atau isi nilainya, berdasarkan berkas definisi tabel.
        
        \item Poin 3 adalah tombol aksi untuk melakukan penyimpanan atau penghapusan pada entri yang saat ini dibuka.
            Tombol simpan akan mengarahkan kembali admin ke halaman daftar entri setelah perubahan disimpan atau
            entri dibuat. Tombol Hapus akan memunculkan modal konfirmasi hapus entri.
    \end{itemize}
    
    
\subsubsection{Pengelola Entitas: Hapus Entri}
    \begin{figure}
        \centering
        \includegraphics{Gambar/mockups/Mockup--Admin - Entity Delete.pdf}
        \caption{Rancangan antarmuka untuk menghapus entri.}
        \label{fig:mockup_admin_entity_delete}
    \end{figure}
    Pada saat tombol hapus pada halaman daftar entri dan \textit{editor} entri diklik, sebuah modal konfirmasi
    penghapusan akan dimunculkan oleh sistem. Rancangan modal tersebut dapat dilihat pada
    \ref{fig:mockup_admin_entity_delete}. Secara garis besar, informasi yang dimunculkan pada modal
    (poin 1) adalah kolom yang ditampilkan pada halaman \textit{editor}. Informasi tersebut ditampilkan 
    untuk menyakinkan admin bahwa admin akan menghapus entri yang tepat.
    
    Jika modal muncul pada halaman \textit{editor}, maka setelah aksi penghapusan berhasil dilakukan,
    admin akan diarahkan menuju halaman daftar entri. Namun jika modal muncul pada halaman entri,
    admin tidak akan diarahkan kemanapun.
    
\subsection{Rancangan Antarmuka untuk Dosen Pengawas}
    % Jelasin kalo ini buat di proyektor
    Antarmuka untuk dosen pengawas akan bertanggung jawab untuk membuka dan menutup lembar jawab,
    serta menampilkan informasi ujian di proyektor tiap ruangan. Tampilan ini mungkin akan 
    berbeda-beda pada tiap ruangan, tergantung dari ujian yang akan dilaksanakan pada ruangan
    tersebut. Karena kendala teknis, rencananya alur untuk menuju tampilan layar akan melalui otentikasi
    terlebih dahulu. 
    
\subsubsection{Otentikasi}
    % Screen login by IP
    \begin{figure}
        \centering
        \includegraphics[width=0.7\paperwidth]{Gambar/mockups/Mockup--DosenPengawas - Login.pdf}
        \caption{Rancangan antarmuka untuk otentikasi, dengan menekankan bagian tertentu.}
        \label{fig:mockup_dosen_login}
    \end{figure}
    Tahap pertama untuk dapat mengakses layar timer, pengguna harus melakukan login terlebih dahulu.
    Serupa dengan halaman Login yang digunakan untuk otentikasi Admin pada Gambar
    \ref{fig:mockup_admin_login}, otentikasi untuk pengguna jenis ini dipertegas dengan penggunaan
    tombol IP Login yang ditunjukan pada poin 1 di Gambar \ref{fig:mockup_dosen_login}.
    
    Tombol IP Login pada Gambar \ref{fig:mockup_dosen_login} direncanakan akan langsung mengarahkan 
    pengguna ke halaman peta ruangan ujian dan timer ujian jika berhasil. IP dari komputer yang
    digunakan untuk mengakses halaman ini harus diregistrasikan ke sistem untuk proses otentikasi
    dapat berhasil. Jika otentikasi gagal, halaman ini harus menampilkan pesan kesalahan yang
    mendeskripsikan bahwa IP komputer tidak teregistrasi.
    
    % Screen Seatmap
\subsubsection{Peta Ruangan}
    \begin{figure}
        \centering
        \includegraphics[width=0.7\paperwidth]{Gambar/mockups/Mockup--DosenPengawas - Seatmap.pdf}
        \caption{Tampilan rancangan antarmuka untuk halaman Peta Ruangan Ujian}
        \label{fig:mockup_dosen_seatmap}
    \end{figure}
    Setelah otentikasi berhasil, dosen pengawas akan diarahkan menuju halaman peta tempat duduk ruangan
    ujian. Rancangan halaman informasi peta ruangan dapat dilihat pada Gambar 
    \ref{fig:mockup_dosen_seatmap}. Pada halaman ini, informasi tentang nomor ruangan dan peta
    tempat duduk ruangan tersebut akan ditampilkan secara diagram.
    
    Bagian-bagian yang ditunjukkan oleh poin pada Gambar \ref{fig:mockup_dosen_seatmap} dijelaskan sebagai
    berikut
    \begin{itemize}
        \item Bagian yang ditunjukan oleh poin 1 dan poin 2 pada Gambar \ref{fig:mockup_dosen_seatmap}
            adalah nagivasi untuk menuju halaman timer dan peta ruangan. Navigasi ditambahkan untuk 
            memudahkan dosen untuk melihat ujian yang akan berjalan dan peta ruangan yang ada saat ini.
            Saat pengguna pindah ke \textit{tab} lain, halaman tidak akan disegarkan atau mengalami pemuatan
            ulang. 
            
        \item Poin 3 dan 4 masing-masing menunjukkan informasi nama ruangan, dan peta tempat duduk dari
            ruangan tersebut. Poin 3 akan menunjukan nama ruangan yang ada dan poin
            4 menunjukkan peta ruangan ujian. Peta ini dirancang untuk membantu peserta untuk mencari
            tempat duduk mereka. Sistem akan menampilkan tempat duduk tempat komputer yang terdaftar
            pada sistem. Informasi tentang posisi komputer rencananya disimpan pada database.
    \end{itemize}
    % Screen Timer
\subsubsection{Timer Ujian}
    \begin{figure}
        \centering
        \includegraphics[width=0.75\paperwidth]{Gambar/mockups/Mockup--DosenPengawas - Timer.pdf}
        \caption{Tampilan rancangan antarmuka untuk halaman Timer.}
        \label{fig:mockup_dosen_timer}
    \end{figure}
    
    \begin{figure}
        \centering
        \includegraphics[width=0.75\paperwidth]{Gambar/mockups/Mockup--DosenPengawas - Timer + Overtime.pdf}
        \caption{Tampilan rancangan antarmuka untuk \textit{Overtime} Ujian}
        \label{fig:mockup_dosen_overtime}
    \end{figure}
    
    Rancangan halaman timer ujian dapat diperhatikan pada Gambar \ref{fig:mockup_dosen_timer}.
    Secara keseluruhan, halaman timer akan menampilkan informasi ujian singkat, waktu tersisa
    dan tombol-tombol alat untuk memanipulasi waktu ujian.
    
    Mengikuti rancangan antarmuka Peta Ruangan, halaman timer memiliki bagian navigasi seperti
    yang ditunjukan pada poin 1 dan 2 pada Gambar \ref{fig:mockup_dosen_timer}. Karena saat ini
    pengguna sedang membuka \textit{tab} Timer, maka kepala \textit{tab} yang aktif menunjukkan
    kepala \textit{tab} timer. Dosen pengawas dapat melihat kedua informasi tersebut dengan mudah.
    Bagian pada rancangan tampilan tersebut dideskripsikan sebagai berikut:
    
    \begin{itemize}
       % poin 3
        \item Bagian yang ditunjukkan pada poin 3 adalah informasi ujian secara singkat. Informasi yang ditunjukkan
            berupa kode mata kuliah dan \textit{shift}nya. Informasi ini nantinya dapat digunakan untuk membedakan
            ujian satu dengan yang lainnya pada tampilan multiujian.
            
        % poin 4
        \item Bagian yang ditunjukkan pada poin 4 adalah jumlah durasi waktu yang diberikan untuk ujian ini. Durasi
            lamanya ujian ini akan diambil dari basis data dan tidak akan berkurang mengikuti tampilan timer.
        
        % poin 5
        \item Bagian yang ditunjukan oleh poin 5 adalah tampilan timer. Timer ini akan memiliki nilai dasar total
            durasi dari ujian. Timer akan mulai berjalan saat kontrol yang ditunjukan pada poin 6 mulai digunakkan.
            Pada saat timer mencapai nilai nol, maka tampilan tersebut rencananya akan di tahan beberapa menit
            untuk memastikan dosen pengawas dapat menambahkan waktu lebih sebelum lembar jawab tidak dapat diubah
            kembali.
        
        % poin 6
        \item Poin 6 menunjukkan bagian kontrol dari timer ujian. Kontrol rencananya terdiri dari tombol aksi mulai
            berhenti, dan setel ulang. Tombol-tombol ini rencananya akan langusng melakukan panggilan API
            ke sistem back-end.
        
        % poin 7
        \item Poin 7 menunjukkan tombol tambahan \textit{overtime}. Tombol ini digunakan untuk menambahkan waktu
            ujian jika sewaktu-waktu dibutuhkan. Tombol akan menampilkan sebuah \textit{modal} yang menyediakan
            sebuah formulir sederhana, menanyakan berapa banyak waktu yang ingin ditambahkan. Modal tersebut
            dapat dilihat pada Gambar \ref{fig:mockup_dosen_overtime}. Modal akan menampilan informasi
            singkat ujian (poin 9), banyak waktu yang diinginkan (poin 10), tombol cepat untuk menambahkan waktu
            (poin 11), dan tombol \textit{submit} dan batalkan (poin 12).
        
        % poin 8
        \item Poin 8 adalah contoh timer lain yang ditunjukkan jika terdapat ujian lain yang akan diadakan pada
            ruangan yang sama. Tampilan akan memiliki bentuk tataletak yang sama, sehingga yang menjadi pembeda
            hanyalah bagian kepala (ditunjukkan pada poin 3).
    \end{itemize}
    
    
\subsection{Rancangan Antarmuka Tambahan}
    Rancangan antarmuka tambahan ini muncul dari kebutuhan untuk sistem mengirimkan laporan
    ujian dengan tambahan halaman pengunduhan. Oleh karena itu, pada bagian ini akan dibahas rancangan 
    antarmuka untuk email dan halaman pengunduhannya.
    
    % Email
\subsubsection{Email Laporan}
    \begin{figure}
        \centering
        \includegraphics[width=0.7\paperwidth]{Gambar/mockups/Mockup--Tambahan.pdf}
        \caption{Tampilan rancangan antarmuka untuk email laporan ujian}
        \label{fig:mockup_addition_email}
    \end{figure}
    Email yang akan dikirimkan sebagai laporan akan dibuat dengan teknologi HTML, seperti pada umumnya.
    Email akan berisi sebuah tautan menuju halaman untuk mengunduh berkas jawaban ujian. Email juga
    rencananya akan berisi beberapa informasi krusial tentang masa hidup tautan tersebut sehingga
    penerima laporan tidak merasa kebingungan jika tautan tidak dapat diakses.
    
    Rancangan untuk tampilan email dapat dilihat pada Gambar \ref{fig:mockup_addition_email}. Seperti yang
    ditunjukkan pada gambar, email akan memiliki sejumlah instruksi untuk mengunduh berkas jawaban, informasi
    mata kuliah ujian (Poin 1 dari Gambar \ref{fig:mockup_addition_email}), tautan untuk mengunduh berkas
    jawaban tersebut (Poin 2 dari Gambar \ref{fig:mockup_addition_email}), dan informasi kapan tautan
    tersebut akan kadaluarsa (Poin 3 dari Gambar \ref{fig:mockup_addition_email}).
    
    % Autonomus download
\subsubsection{Halaman Pengunduhan Berkas Jawaban Ujian}
    Tautan yang dikirim lewat email akan mengarah pada halaman ini. Informasi singkat tentang ujian
    akan ditampilkan pada halaman ini sebelum akhirnya penerima laporan dapat melakukan pengunduhan hasil
    jawaban tersebut. Halaman ini direncanakan untuk disediakan oleh sistem Oxam yang dapat diakses
    via internet.
    
    \begin{figure}
        \centering
        \includegraphics[width=0.7\paperwidth]{Gambar/mockups/Mockup--Tambahan - Exam-extractor.pdf}
        \caption{Tampilan rancangan antarmuka untuk halaman pengunduhan berkas jawaban ujian.}
        \label{fig:mockup_addition_downloader}
    \end{figure}
    Rancangan halaman pengunduhan dapat dilihat pada Gambar \ref{fig:mockup_addition_downloader}.
    Bagian yang ditunjukkan oleh Poin 1 pada Gambar \ref{fig:mockup_addition_downloader} adalah 
    informasi ujian yang disajikan dalam bentuk tabel. Tabel ini nantinya akan berisi semua
    informasi dari entitas ujian, dengan penyesuaian tertentu. Lalu bagian yang ditunjukkan pada
    poin 2 adalah tombol unduh jawaban ujian. Tombol ini akan memaksa peramban untuk melakukan
    pengunduhan berkas jawaban ke komputer penerima laporan.
    
\section{Perancangan Sistem Backend}
    Sesuai dengan perannya, sistem backend akan bertanggung jawab menangani bagian-bagian yang hanya 
    ditangani pada sisi peladen. Tanggung jawab tersebut mulai dari melakukan pengolahan data,
    pengelolaan basis data, serta menyediakan API yang digunakan oleh sistem frontend untuk
    berkomunikasi.
    
    Pada bagian ini akan dijelaskan perancangan sistem untuk subsistem backend. Penjelasan tersebut
    dimulai dari pembahasan perancangan basis data, perancangan REST API 
    serta desain kelas dari subsistem ini.
    
\subsection{Racangan Basis Data}
    \begin{sidewaysfigure}
        \centering
        \includegraphics[width=1\paperwidth]{Gambar/erd-rev-b.pdf}
        \caption{Diagram \textit{ERD} untuk sistem aplikasi yang baru.}
        \label{fig:erd_overview}
    \end{sidewaysfigure}
    
    Perancangan basis data dimulai dengan merumuskan entitas yang dibutuhkan dan relasinya
    dengan entitas lainnya. Rumusan tersebut direpresentasikan dalam bentuk diagram relasi
    entitas atau ERD. Diagram ERD untuk aplikasi ini dapat dilihat pada Gambar \ref{fig:erd_overview}.
    Berdasarkan kebutuhan yang dianalisis pada bab sebelumnya, sistem membutuhkan empat belas
    entitas yang bertanggung jawab untuk merepresentasikan struktur data yang digunakan.
    Setiap entitas yang dibuat akan dibuat ulang representasinya dalam bentuk kelas pada
    sistem PHP dengan harapan membantu penanganan dan perancangan sistem REST API.
    
    Pembahasan akan dilanjutkan dengan menjelaskan satu per satu entitas yang ditunjukan
    pada diagram ERD. Pembahasan akan dimulai dengan entitas yang penulis kategorikan sebagai
    esensial untuk sistem terlebih dahulu, lalu dilanjutkan dengan entitas yang muncul dari
    analisis dari bab sebelumnya.
    
\subsubsection{User}
    % TODO: Figure mini user?.
    Entitas \texttt{User} merepresentasikan pengguna aplikasi. Entitas ini akan menyimpan informasi
    khusus seperti nama pengguna (\texttt{username}), kata sandi (\texttt{password}) dan email.
    Kolom \texttt{password} pada entitas ini akan di\textit{hash} menggunakan algoritma Blowfish
    dengan bantuan \textit{library} dari PHP dengan alasan keamanan.
    
    Entitas ini dinilai esensial karena entitas ini digunakan untuk melakukan otentikasi
    menuju panel admin. Entitas ini akan memiliki perlakuan khusus pada saat data
    dari entitas ini ditransmisikan ke sistem frontend. Otentikasi ini penting untuk
    mengkhususkan peran dari setiap pengguna.

\subsubsection{ACL dan ACLItem}
    \begin{figure}[H]
        \centering
        \includegraphics[width=0.75\paperwidth]{Gambar/erd-details/ERD--New - ACL & ACLItem.pdf}
        \caption{Potongan diagram entitas untuk \texttt{ACL} dan \texttt{ACLItem}.}
        \label{fig:erd_acl-aclitem}
    \end{figure}
    
    Entitas \texttt{ACL} dan \texttt{ACLItem} adalah entitas yang menampung informasi 
    tentang peran dan izin untuk
    setiap pengguna, diperjelas pada Gambar \ref{fig:erd_acl-aclitem}. 
    \texttt{ACL} menyimpan grup utama dari \texttt{ACLItem}, atau dapat kita 
    sebut sebagai peran.
    Sedangkan \texttt{ACLItem} menyimpan informasi izin untuk setiap nama 
    kasus (\texttt{codename}) yang diperbolehkan.
    
    \begin{table}[H]
        \centering
        \begin{tabular}{|l|l|l|}
        \hline
        Label & Deskripsi & Nilai Biner \\ \hline
        C     & Buat      & \texttt{0b0001}        \\ \hline
        R     & Baca      & \texttt{0b0010}        \\ \hline
        U     & Perbarui  & \texttt{0b0100}        \\ \hline
        D     & Hapus     & \texttt{0b1000}        \\ \hline
        \end{tabular}
        \caption{Tabel representasi nilai biner pada kolom \texttt{permission} di entitas \texttt{ACLItem}}
        \label{tab:aclitem_level}
    \end{table}
        
    
    Perizinan direpresentasikan dengan menggunakan sistem biner yang disimpan pada database sebagai angka.
    Representasi biner ini terdiri dari label CRUD, dengan C adalah Buat (\textit{Create}), R adalah
    Baca (\textit{Read}), U adalah Perbaharui (\textit{Update}) dan D adalah Hapus (\textit{Delete}).
    Label tersebut kemudian diberikan tempat khusus pada bitstring sebelum akhirnya dikonversi
    sebagai angka. Nilai-nilai tersebut dapat dilihat pada tabel \ref{tab:aclitem_level}.
    
    
    Sebagai contoh, seorang pengguna dapat \textbf{membuat}, \textbf{membaca}, \textbf{memperbarui}
    namun tidak dapat menghapus sebuah entri. Nilai dari tabel \ref{tab:aclitem_level} digabungkan
    dengan operator \texttt{OR} untuk setiap izin yang diberikan.
    Dengan begitu, nilai \texttt{permission} yang diberikan adalah sebagai berikut:
    
    \begin{subequations}
        \begin{align}
            C \vee R \vee U \vee 0b0000 &= K\\
            0b0001 \vee 0b0010 \vee 0b0100 \vee 0b0000 &= K \\
            K &= 0b0111 \\
            K &= 7
        \end{align}
    \end{subequations}
    
    Hasil kalkulasi kode izin tersebut akan disimpan sebagai 7 pada database sebagai representasi
    kode izin \texttt{0b000}. 
    
    Pengecekan izin \texttt{permission} dilakukan dengan memanfaatkan operator \texttt{AND} antara 
    nilai pada kolom \texttt{permission} dengan nilai \texttt{permission} yang diekspektasikan, lalu
    dibandingkan dengan nilai ekspektasi.
    Sebagai contoh, sistem perlu mengecek apakah pengguna dapat membuat sebuah entri. 
    Sistem dapat melakukan pengecekkan dengan melakukan operasi \texttt{AND} pada nilai 
    \texttt{permission} yang ada saat ini (\texttt{0111}) dengan nilai \texttt{permission} yang
    diekspektasi (\texttt{0010}), dan hasilnya dibandingkan dengan hasil ekspektasi. Kalkulasi
    yang dilakukan dapat dilihat pada simulasi ekuasi berikut:
    
    \begin{subequations}
        \begin{align}
            K \wedge C &= C \\
            0b0111 \wedge C &= C \\
            0b0111 \wedge 0b0001 &= 0b0001 \\
            0b0001 &= 0b0001 \\
            \text{True}
        \end{align}
    \end{subequations}
    
    Sedangkan jika pengecekan apakah seorang pengguna dapat melakukan penghapusan terhadap sebuah entri,
    perhitungan perizinannya menjadi:
    
    \begin{subequations}
        \begin{align}
            K \wedge D &= D \\
            0b0111 \wedge D &= D \\
            0b0111 \wedge 0b1000 &= 0b1000 \\
            0b0000 &= 0b1000 \\
            \text{False}
        \end{align}
    \end{subequations}
    
    Hasil dari ekuasi tersebut kemudian dapat digunakan untuk menentukan apakah seorang pengguna dapat
    melakukan aksi yang diminta atau tidak.
    
    Operasi ini terinspirasi dari \texttt{chmod} dari Linux\footnote{Lihat https://linux.die.net/man/1/chmod}.
    \texttt{chmod} menerima input berupa tiga angka yang merepresentasikan level akses pada \textit{resource}
    tertentu pada sistem. Tiga angka tersebut kemudian diubah menjadi biner untuk dilihat apakah
    seorang pengguna dapat melakukan aksi tertentu. Pendekatan ini kemudian diadaptasi pada sistem ini
    dengan menggunakan level akses yang berbeda (CRUD, dibanding RWX). Dengan begitu pengaturan dan
    pengelolaan resources pada entitas dapat lebih terkontrol dengan lebih fleksibel.
    
\subsubsection{IPLogin}
    \begin{figure}
        \centering
        \includegraphics{Gambar/erd-details/ERD--New - IPLogin.pdf}
        \caption{Potongan diagram entitas untuk \texttt{IPLogin}.}
        \label{fig:erd_iplogin}
    \end{figure}
    Entitas \texttt{IPLogin} digunakan untuk melakukan otentikasi semi-otomatis dengan memanfaatkan IP
    pengguna pada lab, ditautkan dengan sebuah akun pengguna dengan peran yang terbatas (hanya
    dapat melihat dan mengubah entitas tertentu), diperjelas pada Gambar \ref{fig:erd_iplogin}.
    Entitas menyimpan informasi IP, tautan pada pengguna dan lokasi-lokasi, serta catatan khusus
    tentang penautan tersebut. 
    
    Entitas ini memiliki hubungan \textit{many-to-many} dengan entitas \texttt{Location}. Relasi
    ini diharapkan agar Tim Admin dapat melakukan pengaturan komputer master untuk melihat 
    seluruh ujian yang aktif pada ruangan tertentu tanpa harus membuatkan akun khusus 
    tertentu pada sistem. Tim Admin dapat dengan mudah menautkan sejumlah ruangan yang diinginkan
    dengan akun yang memiliki peranan terbatas yang sudah ada sebelumnya.
    
    Entitas ini dinilai cukup esensial karena berhubungan dengan sistem otentikasi yang membatasi
    pengguna untuk melakukan aksi tertentu di admin panel.
    
\subsubsection{Location dan Computer}
    \begin{figure}
        \centering
        \includegraphics{Gambar/erd-details/ERD--New - Location & Computer.pdf}
        \caption{Potongan diagram entitas untuk \texttt{Location} dan \texttt{Computer}.}
        \label{fig:erd_location-computer}
    \end{figure}
    
    Entitas \texttt{Location} dan \texttt{Computer} menyimpan informasi tentang lokasi (ruagan) dan
    daftar pada komputer tersebut sebelum nantinya dihubungkan dengan entitas lainnya. Hubungan
    antara kedua entitas tersebut dapat diperhatikan pada potongan diagram entitas di gambar
    \ref{fig:erd_location-computer}.
    
    Entitas \texttt{Location} menyimpan informasi lokasi dan ruangan tempat peserta dapat mengikuti
    ujian. Pada entitas ini, informasi yang disimpan adalah nama ruangan tersebut dan nama lain
    dari ruangan tersebut. Berdasarkan survei lapangan, setiap ruangan memiliki nama lain, sehingga
    kolom \texttt{name\_alias} ditambahkan. Sedangkan nama ruangan disimpan pada kolom \texttt{room\_name}.
    
    Entitas \texttt{Computer} mendefinisikan komputer yang terdapat pada lokasi tersebut. Oleh karena
    itu hubungan yang dimiliki dengan entitas \texttt{Location} adalah \textit{one-to-many}.
    Entitas ini menyimpan informasi berupa nama komputer (\textit{name}), IPnya (\textit{ip}),
    nama FQDN (\textit{Fully Qualified Domain Name}) atau domain dari komputer tersebut (\textit{reverse\_dns}),
    dan detil letak posisi dari komputer tersebut pada peta ruangan (\textit{d\_pos}).
    
    Kolom \textit{d\_pos} digunakan untuk menyimpan informasi letak komputer pada peta dalam bentuk
    JSON. Bentuk format ini digunakan karena data ini digunakan hanya pada sistem frontend. Dengan
    kesepakatan yang diharapkan cukup fleksibel, maka bentuk pemosisian ini akan disimpan dalam 
    bentuk JSON.
    
    Kolom \textit{ip} digunakan untuk menautkan sebuah komputer dengan IP, sehingga tabel otentikasi
    yang digunakan oleh peserta dapat langsung merujuk pada entitas ini. Setiap komputer peserta
    yang ingin digunakan akan didaftarkan pada entitas ini. Untuk saat ini, karena survei lapangan
    menunjukan bahwa versi IP yang digunakan adalah versi 4, maka fokus aplikasi saat ini akan
    menggunakan IPv4.
    
    Selain itu, terdapat kolom \textit{reverse\_dns} yang digunakan untuk menyimpan informasi nama domain dari
    komputer tersebut. Kolom ini diharapkan dapat membantu Tim Admin dan pengembang aplikasi
    berikutnya untuk melakukan \textit{debugging} dan kebutuhan API lainnya. Nama domain ini didapatkan dari
    LDAP yang digunakan oleh Server Windows untuk melakukan auditing objek mereka. Komputer yang
    terhubung dengan Active Directory milik Lab akan terdaftar pada sistem internal LDAP milik Windows
    Server yang nantinya akan diberikan domain khusus untuk komputer tersebut.

\subsubsection{Lecture dan LecturePeriod}

    \begin{figure}
        \centering
        \includegraphics{Gambar/erd-details/ERD--New - Lecture & LecturePeriod.pdf}
        \caption{Potongan diagram entitas untuk \texttt{Lecture} dan \texttt{LecturePeriod}}
        \label{fig:erd_lecture-lectureperiod}
    \end{figure}

    Entitas \texttt{Lecture} dan \texttt{LecturePeriod} menyimpan informasi tentang mata kuliah
    dan periode tahun ajar mata kuliah bersangkutan pada ujian tertentu. Hubungan antara kedua
    entitas tersebut terhadap entitas \texttt{Exam} dapat dilihat pada potongan diagram entitas di 
    gambar \ref{fig:erd_lecture-lectureperiod}.  Entitas \texttt{Lecture} menyimpan informasi nama
    mata kuliah beserta kodenya, dan \texttt{LecturePeriod} menyimpan informasi tahun ajaran
    dalam bentuk kode.
    
    \begin{table}[]
        \centering
        \begin{tabular}{|l|l|}
        \hline
        Nilai & Tipe Semester \\ \hline
        1     & Ganjil        \\ \hline
        2     & Genap         \\ \hline
        3     & Pendek        \\ \hline
        \end{tabular}
        \caption{Definisi tipe semester dengan nilainya untuk kolom
            \texttt{period\_code} pada entitas \texttt{LecturePeriod}}
        \label{tab:lecture-periode}
    \end{table}
    
    Kode tersebut terdiri dari lima digit angka yang terdiri dari tahun ajar dan tipe semester
    yang sedang berjalan. Empat digit pertama adalah tahun ajar, dan satu digit terakhir adalah
    tipe semester. Tipe semester dan nilai representasinya dapat dilihat lebih jelas pada tabel
    \ref{tab:lecture-periode}.
    
    Sebagai contoh, semester ganjil pada tahun ajar 2016-2017 direpresentasikan sebagai
    \texttt{20161}. Empat digit pertama diambil dari tahun ajar pada tahun dimulai, 2016. Lalu
    berdasarkan tabel \ref{tab:lecture-periode}, semester ganjil memiliki nilai representasi
    1, sehingga kode untuk tahun ajar yang disimpan pada kolom \textit{period\_code} 
    adalah \texttt{20161}.
    
\subsubsection{Participant}
    \begin{figure}
        \centering
        \includegraphics{Gambar/erd-details/ERD--New - Participant.pdf}
        \caption{Potongan diagram entitas untuk \texttt{Participant}}
        \label{fig:erd_participant}
    \end{figure}
    
    Entitas \texttt{Participant} digunakan untuk menyimpan informasi tentang peserta
    ujian di lab. Seperti pada potongan diagram relasi entitas pada gambar \ref{fig:erd_participant},
    entitas ini hanya memiliki dua kolom utama yang digunakan untuk merepresentasikan
    seorang peserta. Kolom \texttt{username} untuk menyimpan informasi username pada sistem
    dan \texttt{npm} untuk menyimpan informasi NPM dari peserta.
    
    Kolom \texttt{npm} digunakan untuk menyimpan string \textit{literal} dari nomor pokok mahasiswa
    (NPM). Karena NPM mahasiswa yang aktif pada saat penelitian ini dibuat terdapat dua standar, maka
    untuk memperingan kinerna basis data, string \textit{literal} NPM ikut disimpan.
    
    Sedangkan kolom \texttt{username} digunakan untuk menyimpan username yang biasa digunakan
    di lab untuk login. Kolom ini akan berguna untuk membantu sistem mebuat \textit{script batch}
    pada tahap pemberian akses ke berkas bantuan ujian dan \textit{workspace} peserta ujian. 
    Pemberian akses dengan script membutuhkan nama pengguna yang digunakan untuk login.
    
\subsubsection{AnswerSlot}
    Entitas \texttt{AnswerSlot} digunakan untuk menyimpan informasi tentang slot jawaban 
    pada ujian tertentu. Entitas ini digunakan sebagai panduan untuk sistem memberikan
    informasi tentang format penamaan file, disimpan pada kolom \texttt{format}. 
    Seperti yang terlihat pada
    diagram entitas pada \ref{fig:erd_overview}, entitas ini akan terhubung secara 
    \textit{many-to-one} terhadap entitas \texttt{exam} dan \textit{one to many} pada entitas
    \texttt{Submission}.
    
    Jawaban yang dikirimkan oleh peserta nantinya akan disimpan pada entitas \texttt{Submission}
    dengan bantuan referensi format dari \texttt{AnswerSlot}. Setiap jawaban yang dikumpulkan akan dicek
    formatnya berdasarkan format yang diberikan pada entitas \textit{AnswerSlot}.

\subsubsection{ExamReport}
    Entitas \texttt{ExamReport} digunakan untuk membantu sistem menjadwalkan pengiriman laporan
    ujian. Laporan ini nantinya direncanakan untuk dikirimkan via email untuk mengatisipasi email
    diblokir oleh penyedia layanan email karena kode Java dan berkas \texttt{.class}nya dianggap 
    sebagai virus.
    
    Seperti yang ditunjukkan pada diagram entitas pada gambar \ref{fig:erd_overview}, entitas 
    \texttt{ExamReport} menyimpan beberapa kolom yang digunakan untuk membantu penjadwalan.
    Kolom pertama adalah \texttt{sent\_on} yang digunakan untuk menandakan apakah ujian ini
    telah dikirimkan pada email (\texttt{to(s)}) ini. Kolom \texttt{valid\_until} digunakan
    untuk menyimpan informasi kapan \texttt{token} akan kadaluarsa (\textit{expired}).
    
    Penggunaan token pada kasus ini dengan harapan alamat link yang dikirim oleh sistem
    ke dosen dapat dengan aman di akses. Kode token akan terdiri dari 16 byte acak yang
    dikonversi menjadi 32 digit heksadesimal string. Token ini diharapkan 
    dibangkitkan dengan standar pengacakan untuk kriptografi.  Pengacakan dengan standar 
    tersebut diharapkan dapat memperkecil faktor kemungkinan serangan prediksi 
    kode token (\textit{guessing attack}).

\subsubsection{Exam}
    \begin{figure}
        \centering
        \includegraphics[width=0.75\paperwidth]{Gambar/erd-details/ERD--New - Exam.pdf}
        \caption{Potongan diagram entitas untuk \texttt{Exam}.}
        \label{fig:erd_exam}
    \end{figure}
    Entitas berikutnya yang akan dibahas adalah entitas \texttt{Exam}, dapat dilihat pada potongan
    gambar \ref{fig:erd_exam}. Entitas ini menyimpan informasi utama untuk ujian dan jadwalnya.
    Informasi jadwal disimpan pada kolom dengan \textit{prefix} \texttt{time\_}. Informasi dasar
    dari ujian tersebut disimpan dengan relasi langsung ke tabel lain seperti \texttt{Lecture},
    \texttt{LecturePeriod}, \texttt{AnswerSlot}, dan seterusnya. 
    
    Informasi yang cukup penting pada entitas ini terdapat pada kolom \texttt{uniqcode}. Kolom
    ini menyimpan informasi kode unik dari sebuah ujian yang kita gunakan untuk menyimpan informasi
    nama folder berkas ujian. Kode ini diharapkan terdiri dengan byte acak yang dibangkitkan dengan
    standar pengacakan untuk kriptografi. Pengacakan dengan standar tersebut diharapkan dapat
    memperkecil faktor kemungkinan serangan prediksi kode unik (\textit{guessing attack}).
    
    Pengecekan ujian yang aktif dilakukan dengan melakukan pengecekan pada kolom 
    \texttt{time\_start} dan \texttt{time\_ended}. Kolom \texttt{time\_start} menandakan
    informasi tentang kapan ujian \textbf{akan} dimulai. Kolom ini tidak menentukan kapan
    lembar jawaban dapat di\textit{submit} oleh peserta ujian. Sedangkan kolom \texttt{time\_ended}
    menandakan informasi tentang kapan lembar jawaban ditutup.
    
    Untuk membuka lembar jawaban, kolom \texttt{time\_opened} dan \texttt{time\_ended} harus diisi
    dengan informasi tanggal kapan lembar jawaban telah dibuka. Pada keadaan bawaan, normalnya
    kolom-kolom tersebut seharusnya memiliki nilai \texttt{NULL}. Pada saat Dosen Pengawas melakukan
    aksi membuka jawaban, maka kedua kolom tersebut harus diisi sesuai dengan durasi yang diberikan
    pada kolom \texttt{time\_duration}. Pengisian kolom \texttt{time\_opened} dam \texttt{time\_ended}
    diharapkan dapat mengurangi beban sistem untuk melakukan perhitungan secara terus menerus jika
    \texttt{time\_ended} tidak diberikan. Selain itu, dengan mengisi dua kolom tersebut secara
    simultan, diharapkan dapat mengurangi potensi masalah ujian lupa ditutup.
    
\subsubsection{Submission}
    
    \begin{figure}[H]
        \centering
        \includegraphics{Gambar/erd-details/ERD--New - Submission.pdf}
        \caption{Potongan diagram relasi entitas untuk \texttt{Submission}}
        \label{fig:erd_submission}
    \end{figure}

    Entitas \texttt{Submission}, seperti yang dapat dilihat pada potongan diagram relasi entitas 
    \ref{fig:erd_submission}, menampung informasi tentang submisi yang dilakukan oleh
    peserta ujian. Submisi yang dikirimkan direncanakan untuk disimpan dengan nama berkas
    yang acak. Nama berkas tersebut kemudian disimpan pada database sebagai referensi untuk nantinya
    pelaporan yang dibantu oleh entitas \texttt{ExamReport}.
    
    Submisi yang dikirimkan oleh peserta akan dicek formatnya dengan bantuan dari entitas 
    \texttt{AnswerSlot}. Jika nama berkas sudah sesuai dengan format, maka data akan dimasukan
    pada entitas ini.
    
\subsubsection{Notification}
    Entitas \texttt{Notification} menyimpan informasi tentang notifikasi yang diberikan
    pada peserta. Karena beragam notifikasi dapat diberikan pada peserta, seperti informasi tentang
    kredensial untuk masuk ke sistem judge atau basis data; atau informasi tentang ralat soal.
    Oleh karena itu hubungan yang dimiliki entitas ini dengan entitas peserta adalah
    \textit{many-to-many}.
    
    Seperti pada diagram relasi entitas pada gambar \ref{fig:erd_overview}, entitas 
    \texttt{Notification} memiliki beberapa kolom untuk menyimpan isi notifikasi 
    (\texttt{description}) dan judul (\texttt{title}). Kolom lainnya, seperti \texttt{type} digunakan
    untuk memasukan notifikasi terhadap grup tertentu. Kolom yang terakhir adalah 
    kolom \texttt{extras}. Semua informasi yang tidak dapat direpresentasikan pada kolom akan
    disimpan dalam bentuk JSON pada kolom ini. Kolom ini disediakan karna kebutuhan fleksibilitas
    notifikasi yang mungkin akan muncul di masa yang akan datang.

\subsection{Rancangan REST API}
    REST API dirancang dengan pola yang seragam pada beberapa \textit{endpoint} besar, sehingga
    penggunaan API dapat lebih optimal diaplikasikan dengan metode abstraksi API class pada sistem
    frontend.
    
    Perancangan REST API dimulai dengan merencanakan metode otentikasi yang akan digunakan untuk
    setiap permintaan. Lalu pembahsan dilanjutkan dengan desain URL yang digunakan dan tujuannya.
    
    % sistem otentikasi
\subsubsection{Otentikasi API}
    Sesuai dengan kebutuhan, Otentikasi API terbagi menjadi dua bagian besar berdasarkan penggunaannya.
    Otentikasi API yang pertama dilakukan dengan menggunakan IP, sedangkan yang kedua, otentikasi
    dilakukan dengan menggunakan token.
    %TODO: MORE
    
    Otentikasi dengan IP diperuntukkan untuk API yang berhubungan dengan ujian pada lokasi yang telah
    didefinisikan sebelumnya. IP digunakan karena sistem sebelumnya menggunakan cara ini dan telah
    terbukti di "medan perang" bahwa cara ini sudah sangat efektif. Pengguna tidak perlu melakukan
    otentikasi secara manual dengan memasukkan kredensial khusus untuk sistem. Otentikasi dengan IP
    juga tidak memaksa klien untuk melakukan refresh secara terus menerus untuk memperbarui 
    \textit{session} mereka, seperti pada aplikasi sebelumnya.
    
    Otentikasi dengan token dilakukan dengan melakukan otentikasi manual hingga sistem memberikan
    sebuah string khusus yang digunakan sebagai token untuk setiap permintaan. Otentikasi
    manual tersebut dapat berupa memasukan kredensial login, atau dengan token khusus yang sebelumnya
    diberikan pada email. Token ini nantinya dapat digunakan untuk mengakses berbagai 
    \textit{resources} yang backend sediakan. Seperti tombol unduh pada halaman laporan, atau
    melakukan CRUD pada entitas tertentu.
    Token yang digunakan pada sistem normalnya akan menggunakan JSON Web Token atau JWT, dengan bantuan 
    \textit{library} dari backend. Rencananya JWT ini akan menggunakan algoritma kunci asimetrik,
    sebagai standar keamanan.
    
    % desail url berdasarkan rest API
\subsubsection{Desain \textit{Endpoint}}
    REST API dibagi menjadi empat bagian besar berdasarkan kebutuhannya. Bagian-bagian tersebut
    memiliki teknik otentikasi yang berbeda-beda. Sebagai contoh untuk endpoint otentikasi
    tidak akan memiliki otentikasi sama sekali, atau, otentikasi Ujian dilakukan secara tidak
    langsung dengan menggunakan IP. 
    
    Seluruh endpoint akan diprefix dengan \texttt{/api/v1/} untuk memberikan fleksibilitas di masa
    yang akan datang pada saat API akan diaplikasikan teknik \textit{versioning}. Selain itu,
    dengan desain url seperti ini, pengembang dapat mengetahui versi yang mereka gunakan pada
    API tersebut.
    
    Pembahasan akan dimulai dengan endpoint untuk otentikasi
    terlebih dahulu, lalu dilanjutkan dengan API ujian, API admin, dan API lainnya.
    
    % Bagian Otentiksi
\subsubsection{Desain \textit{Endpoint} Otentikasi}
    Endpoint untuk otentikasi dapat di akses pada \texttt{system}. Endpoint ini akan bertugas untuk
    menangani segala hal yang berhubungan dengan otentikasi. Mulai dari melakukan pengecekan
    login (\texttt{system/auth/login}), mengambil informasi pengguna yang sedang masuk 
    (\texttt{system/user/me}), hingga membantu untuk menangani 
    \textit{IPLogin} (\texttt{system/auth/iplogin}). 
    Endpoint ini seharusnya bertanggung jawab untuk menerbitkan token JWT
    berdasarkan informasi pengguna terkait yang terekan pada basisdata.
    
    Terdapat satu buah pengecualian untuk otentikasi ujian. Otentikasi ujian dilakukan secara tidak
    langsung dengan melakukan pengecekan IP langsung di level endpoint mereka. Endpoint otentikasi
    tidak akan menerbitkan token untuk peserta ujian, karena peserta ujian dapat langsung
    mengakses endpoint \texttt{exam} dengan catatan. IP komputer harus sudah teregistrasi di
    sistem.
    
    % Bagian Exam
\subsubsection{Desain \textit{Endpoint} Ujian}
    Endpoint untuk ujian memiliki prefiks \texttt{exam}. Endpoint ini akan menggunakan otentikasi
    IP dan tidak membutuhkan penggunanya untuk mengirimkan informasi header \textit{Authorization}
    apapun. Endpoint ini akan melayani segala kebutuhan peserta untuk ujian.
    
    % Bagian Admin
\subsubsection{Desain \textit{Endpoint} Admin}
    
    % Bagian Dosen
\subsubsection{Desain \textit{Endpoint} Lainnya}
    
\subsection{Desain Kelas}
    % diagram kelas
    Desain kelas dari aplikasi
    
\section{Perancangan Sistem Frontend}
    % diagram kelas dan file
    
    % Polling pada api exam.
    
    % Jelasin komponennya satu satu
\section{Perancangan Sistem CI/CD dan Unit Testing}
    % melakukan build pada saat commit
    Berdasarkan hasil analisis pada bab sebelumnya, penelitian membutuhkan sebuah sistem yang dapat
    melakukan \textit{build} secara otomatis sebelum dapat di-\textit{deploy} ke peladen produksi.
    Sistem \textit{build} tersebut akan direncanakan untuk diimplementasi pada sistem CI/CD yang
    terintegrasi dengan tempat penyimpanan repositori kode sumber. Kode sumber tersebut nantinya akan
    dilacak oleh sistem repository sehingga pada saat kode mengalami perubahan, sistem CI/CD akan
    secara otomatis berjalan. Sistem ini direncakan untuk melakukan hal berikut secara sekuensial:
    
    \begin{itemize}
        \item Melakukan penyegaran dependensi.
        \item Melakukan \textit{build} dengan level produksi.
        \item Menyimpan hasil \textit{build} pada repositori.
        \item Unggah perubahan pada repositori kembali ke layanan penyimpanan.
        \item Picu sistem \textit{deploy} untuk melakukan \textit{deployment}.
    \end{itemize}
    
    Hal tersebut dilakukan untuk memastikan hasil \textit{build} dapat dikirimkan pada peladen untuk disediakan
    ke pengguna. Sistem \textit{deployment} yang digunakan diimplementasi dengan bantuan script yang dijalankan
    via SSH. \textit{Script} tersebut direncanakan akan melakukan pengunduhan perubahan kode, lalu melakukan
    penyegaran dependensi versi server, dan mengabari kembali sistem CI/CD bahwa \textit{deployment} telah
    berhasil dilakukan.
    
    % Unit testing part
    Selain melakukan \textit{build}, penelitian juga akan menambahkan sistem tes unit otomatis. Sistem tes ini
    direncakan untuk melakukan pengecekan terhadap komponen-komponen krusial pada aplikasi. Karena konsekuensi
    yang ditanggung aplikasi cukup besar, maka sistem tes ditambahkan untuk mencegah kesalahan yang sangat
    fatal. Sistem testing ini direncanakan akan melakukan pengujian pada:
    
    \begin{itemize}
        \item Sistem Otentikasi \\
            Sistem otentikasi diharapkan dapat melakukan penolakan akses tidak berizin pada beberapa sumber daya
            yang dinaungi oleh sistem ini. Salah satu yang menjadi krusial untuk dipastikan adalah
            hanya pengguna dengan peran Admin yang dapat melakukan aksi tulis pada REST API untuk
            Admin.
        \item Penjadwalan Ujian \\
            Fungsi penjadwalan ujian adalah salah satu fungsi yang cukup krusial. Fungsi ini akan bertanggung
            jawab untuk mencegah ujian lain tampil pada ujian lainnya. Jika ujian lain tampil, maka lembar jawab
            untuk ujian tersebut juga akan terbuka, sehingga peserta bisa saja mengunduh jawaban ujian
            dari \textit{shift} sebelumnya. Pada dasarnya, tujuan diadakannya tes unit untuk fungsi ini
            adalah untuk memastikan bahwa fungsi yang memiliki konsekuensi besar dapat berjalan dengan
            semestinya.
    \end{itemize}
    
    % melakukan unit testing pada saat commit
    Sistem tes unit ini direncakan berjalan bersama sistem \textit{build}. Sehingga kode akan di tes setiap kali
    kode baru diunggah ke tempat penyimpanan.