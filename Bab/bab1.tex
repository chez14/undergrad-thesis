%versi 2 (8-10-2016) 
\chapter{Pendahuluan}
\label{chap:intro}
   
\section{Latar Belakang}
\label{sec:label}
Ujian praktik menjadi salah satu syarat untuk memenuhi komponen penilaian
suatu mata kuliah.
Salah satu ujian praktik pada Program Studi Teknik Informatika 
dilaksanakan pada lab komputasi dengan bantuan aplikasi Oxam.
Tim Admin Lab Komputer akan bertugas untuk membantu
pelaksanaan ujian dengan mempersiapkan sistem dan ruangan yang akan digunakan
untuk melaksanakan ujian. Tim Admin yang bertugas akan ditentukan oleh Kepala Lab pada 
saat jadwal ujian telah terbit. Tim Admin yang bertugas kemudian akan diberikan informasi soal ujian
dan berkas-berkas bantuannya untuk melakukan pengaturan sistem dan konfigurasi tempat duduk dan ruangan.
Peserta ujian akan diberi soal ujian melalui aplikasi
Oxam yang berjalan di lab sesuai prosedur dan aturan yang berlaku.

Aplikasi Oxam yang berjalan pada lab saat ini bertugas untuk
membantu mengatur berbagai kebutuhan seperti pengumpulan jawaban, pengacakan
daftar peserta, serta pengarsipan berkas jawaban. Aplikasi yang saat ini
digunakan adalah Oxam versi ke-4 (Gambar \ref{fig:ss-Oxam}). Oxam v4 bekerja dengan meminta
data-data ujian berupa kode matakuliah, tipe ujian, jurusan, jam mulai ujian, daftar
peserta, \textit{slot} tempat duduk yang dapat digunakan, dan daftar nama berkas
yang akan dikumpulkan. Aplikasi Oxam v4 akan secara otomatis membuatkan daftar
tempat duduk peserta dan membuatkan \textit{script} untuk
menyalin berkas ujian ke komputer peserta. Tempat duduk peserta akan diacak dengan
alasan keamanan agar kemungkinan peserta untuk melakukan kerja sama antar peserta lain
menjadi lebih kecil. \textit{Script} yang digunakan untuk menyalin berkas peserta akan
dijalankan pada \textit{server} dengan sistem operasi Windows Server karena komputer peserta
menggunakan sistem operasi Windows. Oxam v4 yang saat ini digunakan berjalan diatas server
berbasis Linux, karena Oxam v4 dibangun pada pemrograman berbasis web.

\begin{figure}
    \centering
    \includegraphics[width=0.7\paperwidth]{Gambar/ss-oxam.png}
    \caption{Tampilan cuplikan layar dari Oxam, aplikasi manajemen ujian di Lab Komputer.}
    \label{fig:ss-Oxam}
\end{figure}

Namun fitur yang terbatas membuat Oxam menjadi tidak efektif untuk menyelesaikan
insiden-insiden khusus. Salah satu masalah yang sering dihadapi adalah
pemindahan posisi peserta ke meja lain saat masalah terjadi. Admin harus
mengubah secara manual entri pada basis data yang bersangkutan, lalu memindahkan
berkas ujian tersebut secara manual ke posisi yang baru. Selain itu dengan
perubahan NPM (Nomor Pokok Mahasiswa) untuk angkatan 2018 dan selanjutnya
membuat sistem Oxam v4 tidak dapat digunakan tanpa harus mengubah NPM
tersebut ke bentuk yang lama. Perubahan tersebut dapat dilihat pada Tabel~\ref{tab:table-npm}. 

\begin{table}[H]
    \centering
    \caption{NPM lama dan NPM baru yang sistem informasi gunakan.}
    \label{tab:table-npm}
    \def\arraystretch{2}
    \begin{tabular}{|c|c|}
        \hline
        \textbf{NPM Lama} & \textbf{NPM Baru} \\
        \hline
        201673\textbf{0011} & 6181601\textbf{011} \\
        \hline
    \end{tabular}
\end{table}

Karena sistem Oxam tertegrasi dengan layanan server lain, maka NPM harus
distandarisasi dengan memetakan NPM ke username. Pemetaan NPM menjadi username
ini menjadi bermasalah karena perbedaan struktur NPM yang berbeda. Perbedaan ini
meliputi seperti, nomor kode jurusan (Informatika adalah 73, saat ini menjadi
618), lalu posisi tahun yang berpindah dan adanya kode reguler (01) dan
non-reguler pada depan nomor urut. Perbedaan ini membuat sistem lama tidak dapat
memetakan NPM baru ke username yang biasanya digunakan oleh sistem yang sudah
ada di lab komputasi saat ini.

Selain itu runtutan kegiatan yang dilakukan pada saat fase persiapan ujian pada
Lab Komputasi dengan aplikasi ini terlalu banyak. Berdasarkan pengalaman,
hal ini menimbulkan beberapa \textit{human error} sebagai berikut:
    \begin{itemize}
        \item Berkas daftar duduk peserta yang tertimpa oleh sesi ujian
            berikutnya. Berkas daftar tempat duduk dibuat menjadi berkas HTML
            yang harus dicetak. Daftar berkas tersebut dapat dilihat pada Gambar
            \ref{fig:ss-folder-gen}.\\
            Jika Admin lupa mencetak atau menyalin berkas tersebut ke komputer
            lokal, Admin tersebut diharuskan untuk menghapus entri ujian
            tersebut, lalu mendaftarkan ulang sesi ujian tersebut beserta dengan
            daftar peserta dan daftar tempat duduk yang digunakan.

        \item Jika Admin melakukan \textit{copy} dengan urutan yang salah, \textit{folder} untuk
            ujian tidak akan terbuat, atau bahkan tidak dapat diakses oleh peserta.
        
        \item Salah memasukkan daftar peserta ujian.
        
        \item Menghapus folder berkas ujian yang lama pada server. Jika petugas
            tersebut lupa, maka konsekuensinya adalah pada saat pengumpulan,
            Admin yang bertugas harus memisahkan berkas ujian lama dan yang baru
            secara manual.
    \end{itemize}

\begin{figure}
    \centering
    \includegraphics[width=0.7\paperwidth]{Gambar/ss-struktur-folder-generator.png}
    \caption{Daftar peserta yang di\textit{generate} oleh Oxam dalam bentuk
    berkas.}
    \label{fig:ss-folder-gen}
\end{figure}

Masalah berikutnya muncul pada saat proses ujian tersebut berjalan. Pertama,
terdapat \textit{bug} waktu ujian telah habis, pada kenyataannya waktu ujian
belum habis. Kedua, \textit{timer} yang digunakan untuk menunjukkan sisa waktu
ujian tidak tersingkronisasi dengan Oxam. Sehingga pada saat timer berbunyi,
tempat pengumpulan tidak langsung tertutup. Ketiga, entri ujian yang sudah
dihapus masih muncul pada tempat pengumpulan. Hal ini biasanya diatasi oleh tim
admin dengan cara mengubah tanggal sesinya ke tahun lalu.

Pada fase pengumpulan berkas jawaban ujian ke dosen koordinator, sistem tidak
secara otomatis mengumpulkan berkas tersebut. Sehingga seringkali Admin yang
bertugas lupa untuk mengirimkan berkas tersebut. Pengumpulan berkas tersebut
seharusnya dikirimkan sesegera mungkin saat ujian sudah selesai. Hal ini
dimaksudkan agar jawaban tidak diubah di kemudian hari tanpa izin.

Selain masalah-masalah pada tiap fase tersebut, masalah lain ada pada sistem itu
sendiri. Oxam menyimpan berkas tanpa mengacak lokasi atau nama berkas jawaban,
diperlihatkan pada Gambar \ref{fig:ss-folder-jawaban}. Hal ini dapat
mempermudah penyerang sistem untuk mengubah berkas jawaban tertentu tanpa
harus bersusah payah.

\begin{figure}[ht]
    \centering
    \includegraphics[width=0.6\paperwidth]{Gambar/ss-struktur-folder-jawaban.png}
    \caption{Struktur folder jawaban pada sistem Oxam.}
    \label{fig:ss-folder-jawaban}
\end{figure}

Pada penelitian ini, akan dibangun ulang aplikasi baru untuk
menyelesaikan masalah-masalah yang muncul pada aplikasi lama dengan
menggunakan \textit{framework Fat-free} dan \textit{React}.

\section{Rumusan Masalah}
\label{sec:rumusan}
Pada skripsi ini, aplikasi akan membantu memecahkan masalah:
\begin{itemize}
    \item Apa saja fitur yang dibutuhkan untuk aplikasi Oxam v5 di lab komputer?
    
    \item Bagaimana mengimplementasikan fitur-fitur untuk aplikasi Oxam v5 di lab komputer?
\end{itemize}

\section{Tujuan}
\label{sec:tujuan}
Tujuan dari skripsi ini adalah sebagai berikut:
\begin{itemize}
    \item Mencari tahu fitur-fitur yang dibutuhkan untuk aplikasi Oxam v5 beserta dengan kebutuhannya.

    \item Melakukan implementasi fitur-fitur yang dibutuhkan dengan bantuan \textit{Framework} dan 
        \textit{Library}.

\end{itemize}

\section{Batasan Masalah}
\label{sec:batasan}
Batasan masalah pada penelitian ini adalah sebagai berikut:
\begin{enumerate}
    \item Implementasi kebutuhan aplikasi Oxam v5 diimplementasi pada server Linux.
    
    \item \textit{Script} yang dihasilkan oleh Oxam v5 harus 
        dapat berjalan pada server dengan sistem operasi Windows Server.
\end{enumerate}

\section{Metodologi}
Metodologi yang dilakukan pada penelitian ini adalah sebagai berikut:
\label{sec:metlit}
    \begin{enumerate}
        \item Studi literatur bahasa dan \textit{framework} Fat-free dan
            \textit{libary} React.js.
        \item Melakukan analisis dengan survei dan menyebar kuisioner untuk menentukan fitur yang diperlukan oleh
            Oxam v5.
		\item Melakukan implementasi fitur-fitur yang dibutuhkan oleh Oxam v5.
	    \item Melakukan \textit{deployment} dan pengujian pada fungsionalitas
	        aplikasi.
        \item Menarik kesimpulan dan saran berdasarkan proses penelitian dan
            pengujian.
        \item Membuat dokumen laporan skripsi berdasarkan penelitian yang telah dilakukan.
    \end{enumerate}

\section{Sistematika Pembahasan}
\label{sec:sispem}

Pembahasan penelitian akan dilakukan secara sistematis dengan detail sebagai
berikut:

\begin{itemize}
    \item Bab 1 Pendahuluan \\
        Berisi latar belakang dibuatnya penelitian aplikasi manajemen ujian di lab komputer, 
        rumusan masalah, tujuan, batasan masalah, metodologi serta sistematika pembahasan penelitian ini.
    
    \item Bab 2 Landasan Teori \\
        Bab ini berisi Pedoman Pelaksanaan Ujian di Lab Komputasi, landasan
        teori dari Aplikasi Berbasis Web, \textit{Framework} dan
        \textit{Library}, REST API, CI/CD serta Docker yang akan menjadi
        landasaan untuk membantu analisis penelitian aplikasi manajemen ujian di
        lab komputer.
        
    \item Bab 3 Analisis \\
        Berisi pembahasan analisa aplikasi Oxam v4 pada lab komputer,
        pelaksanaan ujian, analisa kebutuhan dan fitur aplikasi Oxam berdasarkan
        kuisioner, pemilihan \textit{framework} dan \textit{library}, serta
        Analisis pengguna, serta skenarionya.
        
    \item Bab 4 Perancangan \\
        Pada bab ini akan dijabarkan tentang perancangan aplikasi Oxam v5 yang akan
        diimplementasi untuk membantu memanajemen ujian di lab komputer.
        Perancangan tersebut akan terdiri dari perancangan tampilan antar muka
        untuk peserta, admin, layar proyektor dan lembar jawab. Lalu perancangan
        dilakukan juga untuk basis data, API, juga sistem CI/CD.
    
    \item Bab 5 Implementasi dan Pengujian \\
        Berisi pembahasan implementasi aplikasi Oxam v5 yang telah dirancang dan
        pengujian aplikasi tersebut. Pengujian akan terdiri dari pengujian
        eksperimental dan fungsional.
        
    \item Bab 6 Kesimpulan dan Saran \\
        Berisi kesimpulan dan saran dari penelitian aplikasi Oxam v5 di
        lab komputer berdasarkan perancangan, implementasi dan pengujian yang
        telah dilakukan.
\end{itemize}