%versi 2 (8-10-2016) 
\chapter{Pendahuluan}
\label{chap:intro}
   
\section{Latar Belakang}
\label{sec:label}
Ujian menjadi salah satu syarat yang mutlak untuk memenuhi komponen penilaian suatu mata kuliah. Salah satu bentuk ujian tersebut dilakukan secara praktik. 
Ujian praktik ini dilaksanakan pada lab komputasi dengan bantuan aplikasi. 
Pihak yang bertanggung jawab untuk mempersiapkan ruangan dan sistem adalah \textit{System Administrator} atau Admin. 
Peserta akan diberi soal ujian melalui sistem yang berjalan di lab sesuai prosedur dan aturan yang berlaku.

Ujian pada Lab Komputasi dilakukan dengan bantuan perangkat lunak. Perangkat lunak tersebut membantu mengatur berbagai kebutuhan seperti pengumpulan jawaban, pengacakan daftar peserta, serta pengarsipan berkas jawaban.
Perangkat lunak yang saat ini digunakan bernama Oxam (Gambar \ref{fig:ss-Oxam}).
Oxam bekerja dengan meminta parameter berupa kode matakuliah, tipe ujian, jurusan, jam mulai ujian, daftar peserta, \textit{slot} tempat duduk yang dapat digunakan, dan daftar nama berkas yang akan dikumpulkan. Perangkat lunak Oxam akan secara otomatis membuatkan daftar tempat duduk peserta yang sudah teracak, dan membuatkan \textit{script} untuk menyalin berkas ujian ke komputer peserta.

\begin{figure}
    \centering
    \includegraphics[width=0.7\paperwidth]{Gambar/ss-oxam.png}
    \caption{Tampilan cuplikan layar dari Oxam, aplikasi manajemen ujian di Lab Komputasi.}
    \label{fig:ss-Oxam}
\end{figure}

Namun fitur yang terbatas membuat Oxam menjadi tidak efektif untuk menyelesaikan insiden-insiden khusus. Salah satu masalah yang sering dihadapi adalah perpindahan posisi peserta ke meja lain saat masalah terjadi. Admin harus mengubah secara manual entri pada database yang bersangkutan, lalu memindahkan berkas ujian tersebut secara manual ke posisi yang baru.
Perubahan NPM (Nomor Pokok Mahasiswa) untuk angkatan 2018 dan seterusnya membuat sistem Oxam yang lama tidak dapat digunakan tanpa harus mengubah NPM tersebut ke bentuk yang lama, lihat Tabel \ref{tab:table-npm}. 

\begin{table}[H]
    \centering
    \def\arraystretch{2}
    \begin{tabular}{|c|c|}
        \hline
        \textbf{NPM Lama} & \textbf{NPM Baru} \\
        \hline
        201673\textbf{0011} & 6181601\textbf{011} \\
        \hline
        \multicolumn{2}{|c|}{\textbf{username}} \\
        \hline
        \multicolumn{2}{|c|}{i16\textbf{011}} \\
        \hline
    \end{tabular}
    \caption{NPM lama, baru dan username yang mahasiswa gunakan untuk login.}
    \label{tab:table-npm}
\end{table}

NPM lama memiliki empat komponen besar yang digunakan untuk mengidentifikasi jenis mahasiswa (Tabel \ref{tab:table-npm}, kolom NPM Lama). 
\begin{itemize}
    \item Empat karakter pertama pada NPM adalah informasi tahun mahasiswa tersebut memulai kuliah. Pada contoh tampak angka 2016, yang berarti mahasiswa tersebut memulai perkuliahannya pada tahun 2016 (angkatan 2016). 
    
    \item Dua karakter berikutnya menandakan jurusan (dan fakultas) mahasiswa tersebut belajar. Pada contoh tampak angka 73 yang menandakan bahwa mahasiswa tersebut belajar pada Fakultas Teknologi Informasi dan Sains (7), dan berada pada jurusan Informatika (73). 
    
    \item Satu karakter berikutnya menandakan program yang diambil oleh mahasiswa tersebut. Angka 0 menunjukkan bahwa mahasiswa tersebut mengikuti program Sarjana, seperti pada tabel contoh. Angka 1 menandakan mahasiswa tersebut mengambil program pascasarjana dan angka 2 menunjukkan program doktoral. 
    
    \item Tiga angka berikutnya yang menjadi komponen terakhir dari NPM lama ini adalah nomor urut mahasiswa tersebut. Dengan spesifikasi berikut, berdasarkan tabel contoh, mahasiswa dengan NPM 2016730011 adalah mahasiswa angkatan 2016, mengikuti program sarjana informatika pada Fakultas Teknologi Informasi dan Sains dengan nomor urut 11.
\end{itemize}

NPM baru ini dibuat pada tahun 2018, dan memiliki struktur yang jauh berbeda dari NPM lama. Format NPM tersebut didefinisikan sebagai berikut:

\begin{itemize}
    \item Angka pertama menandakan program jenjang yang diambil oleh mahasiswa tersebut. Angka 5 untuk diploma, 6 untuk sarjana, 8 pascasarjana dan 9 untuk doktoral. 

    \item Dua digit angka berikutnya menunjukkan program studi yang diambil oleh mahasiswa tersebut, pada contoh 18 berarti Informatika. 
    
    \item Dua digit berikutnya menunjukan angka tahun mulai perkuliahan (angkatan) dalam format representasi tahun dengan dua digit. Pada contoh tampak angka 16 yang berarti mahasiswa tersebut adalah angkatan 2016. 
    
    \item Dua digit berikutnya menginformasikan jenis mahasiswa. Pada contoh tampak angka 01 yang menandakan bahwa mahasiswa tersebut berjenis reguler.
    
    \item Tiga digit terakhir menginformasikan nomor urut mahasiswa tersebut.
\end{itemize}

Pada tabel \ref{tab:table-npm} terdapat kolom username yang digunakan untuk menstandarisasi NPM tersebut. Informasi username ini nantinya dimanfaatkan untuk mengintegrasi berbagai macam sistem yang membutuhkan informasi mahasiswa. Digit pertama pada username menginformasikan jurusan mahasiswa tersebut. Huruf i menginformasikan mahasiswa tersebut adalah mahasiswa jurusan Informatika, huruf m untuk matematika, dan huruf f untuk fisika. Dua digit berikutnya menginformasikan tahun angkatan mahasiswa tersebut dalam bentuk representasi tahun dalam dua digit. Pada contoh tampak angka 16, yang menandakan bahwa username ini adalah milik mahasiswa angkatam 2016. Tiga digit terakhir berikutnya menginformasikan nomor urut mahasiswa tersebut.

Karena sistem Oxam terintegrasi dengan layanan server lain, maka NPM tersebut harus distandarisasi agar dapat digunakan oleh sistem. Pemetaan NPM menjadi Username adalah salah satu proses standarisasi tersebut. Pemetaan NPM menjadi username ini menjadi bermasalah karena perbedaan struktur NPM tersebut. Perbedaan ini meliputi seperti, nomor kode jurusan (Informatika adalah 73, saat ini menjadi 618), lalu posisi tahun yang berpindah dan adanya kode reguler (01) dan non-reguler pada depan nomor urut. Perbedaan ini membuat sistem lama tidak dapat memetakan NPM baru ke username yang biasanya digunakan oleh sistem yang sudah ada di lab komputasi saat ini.

Selain itu runtutan kegiatan yang dilakukan pada saat fase persiapan ujian pada Lab Komputasi dengan perangkat lunak ini terlalu banyak. Berdasarkan pengalaman, hal ini menimbulkan banyak sekali \textit{human error}, seperti:
    \begin{itemize}
        \item Salah satu masalah yang muncul adalah berkas daftar duduk peserta yang tertimpa oleh sesi ujian berikutnya, dapat dilihat pada Gambar \ref{fig:ss-folder-gen}.\\
        Jika Admin lupa melakukan penyimpanan, Admin tersebut diharuskan untuk menghapus entri ujian tersebut, lalu mendaftarkan ulang sesi ujian tersebut beserta dengan daftar peserta dan daftar tempat duduk yang digunakan.
        \item Jika Admin melakukan copy dengan urutan yang salah, folder untuk ujian tidak akan terbuat, atau bahkan tidak dapat diakses oleh peserta.
        
    \end{itemize}

\begin{figure}
    \centering
    \includegraphics[width=0.7\paperwidth]{Gambar/ss-struktur-folder-generator.png}
    \caption{Daftar peserta yang di\textit{generate} oleh Oxam dalam bentuk berkas.}
    \label{fig:ss-folder-gen}
\end{figure}

Petugas Admin yang berkewajiban untuk menjaga harus memastikan bahwa kumpulan berkas ujian yang lama harus dihapus terlebih dahulu sebelum memulai sesi ujian yang baru. Jika petugas tersebut lupa, maka konsekuensinya adalah pada saat pengumpulan, Admin yang bertugas harus memisahkan berkas ujian lama dan yang baru secara manual.

Masalah berikutnya muncul pada saat proses ujian tersebut berjalan. Pertama, terdapat \textit{bug} waktu ujian telah habis, pada kenyataannya waktu ujian belum habis. Kedua, \textit{timer} yang digunakan untuk menunjukan sisa waktu ujian tidak tersingkronisasi dengan Oxam. Sehingga pada saat timer berbunyi, tempat pengumpulan tidak langsung tertutup. Ketiga, entri ujian yang sudah dihapus masih muncul pada tempat pengumpulan. Hal ini biasanya Admin tangani dengan cara mengubah tanggal sesinya ke tahun lalu.

Pada fase pengumpulan berkas jawaban ujian ke dosen koordinator, sistem tidak secara otomatis mengumpulkan berkas tersebut. Sehingga seringkali Admin yang bertugas lupa untuk mengirimkan berkas tersebut. Pengumpulan berkas tersebut seharusnya dikirimkan sesegera mungkin saat ujian sudah selesai. Hal ini dimaksudkan agar jawaban tidak diubah di kemudian hari tanpa izin.

Selain masalah-masalah pada tiap fase tersebut, masalah lain ada pada sistem itu sendiri. Oxam menyimpan berkas tanpa mengacak lokasi atau nama berkas jawaban tersebut, diperlihatkan pada Gambar \ref{fig:ss-folder-jawaban}. Hal ini dapat mempermudah penyerang sistem untuk mengubah berkas jawaban tersebut.

\begin{figure}
    \centering
    \includegraphics[width=0.6\paperwidth]{Gambar/ss-struktur-folder-jawaban.png}
    \caption{Struktur folder jawaban pada sistem Oxam.}
    \label{fig:ss-folder-jawaban}
\end{figure}

Pada penelitian ini, diharapkan akan menghasilkan perangkat lunak baru untuk menyelesaikan masalah-masalah yang muncul pada perangkat lunak lama dengan menggunakan \textit{framework Fat-free} dan \textit{React.js}.

\section{Rumusan Masalah}
\label{sec:rumusan}
Pada skripsi ini, aplikasi akan membantu memecahkan masalah:
\begin{itemize}
    \item Apa saja kebutuhan perangkat lunak untuk ujian untuk sistem manajemen ujian di lab komputasi?
    
    \item Bagaimana implementasi pemenuhan kebutuhan perangkat lunak sistem manajemen ujian di lab komputasi?
\end{itemize}

\section{Tujuan}
\label{sec:tujuan}
Tujuan dari skripsi ini adalah sebagai berikut:
\begin{itemize}
    \item Melakukan survei dan analisis untuk mendapatkan daftar kebutuhan perangkat lunak ujian untuk sistem informasi manajemen ujian di lab komputasi.
    \item Pemenuhan kebutuhan diimplementasi dengan membuat ulang perangkat lunak dengan menggunakan \textit{framework}, dengan harapan dapat terus di pelihara oleh tim admin di kemudian hari.
\end{itemize}

\section{Batasan Masalah}
\label{sec:batasan}
Batasan masalah pada penelitian ini adalah sebagai berikut:
\begin{enumerate}
    \item Solusi yang diajukan akan berupa aplikasi pendukung sistem ujian.
  
    \item Aplikasi pendukung ujian akan berjalan pada server berbasis Linux, sehingga dibutuhkannya bantuan untuk mengeksekusi \textit{script batch} pada sistem berbasis Windows.
\end{enumerate}

\section{Metodologi}
Metodologi yang dilakukan pada penelitian ini adalah sebagai berikut:
\label{sec:metlit}
    \begin{enumerate}
        \item Studi literatur bahasa dan \textit{framework} Fat-free dan \textit{libary} React.js.
        \item Melakukan survei sistem dan menyebar kuisioner.
		\item Melakukan perancangan ERD basis data.
		\item Melakukan perancangan tampilan antarmuka.
		\item Mengimplementasi modul/entitas berikut:
		    \begin{itemize}
		        \item Mata Kuliah
		        \item Peserta Ujian
		        \item Ujian (Sesi Ujian)
		        \item Print Daftar Peserta
		        \item Pengumuman untuk Peserta
		    \end{itemize}
	     \item Implementasi Tampilan untuk Peserta dengan detil:
		    \begin{itemize}
		        \item Tempat Pengumpulan
		        \item \textit{Sumary} Ujian
		        \item Bagian informasi/notifikasi
		    \end{itemize}
		\item Implementasi Admin Panel untuk Admin.
	    \item Melakukan \textit{deployment} dan pengujian pada fungsionalitas perangkat lunak.
	    \item Implementasi pengiriman berkas jawaban ujian secara otomatis ke dosen bersangkutan.
        \item Menarik kesimpulan dan saran berdasarkan proses penelitian dan pengujian.
    \end{enumerate}

\section{Sistematika Pembahasan}
\label{sec:sispem}

Pembahasan penelitian akan dilakukan secara sistematis dengan detail sebagai berikut:

\begin{itemize}
    \item Bab 1 Pendahuluan \\
        Berisi latar belakang penelitian, rumusan masalah, tujuan batasan masalah, metodologi dan sistematika pembahasan penelitian.
    
    \item Bab 2 Landasan Teori \\
        Bab ini berisi teori dari Back-end dan Front-end, Framework, REST API, CI/CD dan Docker.
        
    \item Bab 3 Analisis \\
        Berisi pembahasan analisa sistem masa kini, pelaksanaan ujian, kuisioner, analisa kebutuhan dan fitur aplikasi, pemilihan framework dan basis data.
        
        
    \item Bab 4 Perancangan \\
        Berisi rancangan yang dibuat berdasarkan hasil analisis pada bab sebelumnya.
    
    \item Bab 5 Implementasi dan Pengujian \\
        Berisi pembahasan implementasi aplikasi yang dirancang dan pengujian aplikasi tersebut.
        
    \item Bab 6 Kesimpulan dan Saran \\
        Berisi kesimpulan dan saran dari penelitian ini.
\end{itemize}